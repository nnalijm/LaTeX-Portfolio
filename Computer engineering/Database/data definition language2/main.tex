\documentclass{article}

\usepackage{../packages}

\graphicspath{{./figures}}


\begin{document}
\begin{titlepage}
\begin{center}
	\includegraphics[scale=0.7]{logo.png}

	\vspace*{4cm}
	\textbf{Bazy danych\\ Laboratorium}

	\vspace{1.5cm}
	\textit{Zapytania DDL SQL Oracle 2}

	\vspace{1.5cm}
	\textbf{Stanislau Antanovich}\\
	nr. indeksu: 173590\\
	gr. lab: L04

	\vspace{4.5cm}
	\today
\end{center}
\end{titlepage}

\tableofcontents
\listoffigures
\lstlistoflistings

\newpage
 
\section{Wprowadzenie}

\subsection{Cel ćwiczenia}

\subsection{Przygotowanie}

\section{Realizacja}

\begin{enumerate}
\item Tworzenie i wywołanie procedury \textbf{dodaj\_etat} pozwalającej na dodanie nowego etatu

\item Tworzenie i wywołanie procedury \textbf{usun\_pracownika} pozwalającej na usunięcie pracownika na podstawie podanego id(parametr wejściowy)

\item Tworzenie i wywołanie proceduty \textbf{edytuj\_pracownika} pozwalającej na edytowanie kolumn pracownika na podstawie podanych parametrów

\item Modyfikacja procedury \textbf{dodaj\_tekst} w taki sposób, aby uniknąć wystąpienia duplikującego się rekordu.

\item Zaproponowana funkcja \textbf{srednia\_pensja}, która zwraca średnią pensję wszystkich pracowników

\item Zaproponowana funkcja \textbf{zmien\_prowizje}, która umożliwi na podstawie id pracownika zmienić jego prowizję. W przypadku wprowadzonej wartości poniżej zera funkcja ma zgłosić odpowiedni wyjątek

\item Zaproponowana funkcja \textbf{ilosc\_pracownicy\_z\_pensja} umożliwiająca zwrócenie ilości pracowników, których pensja mieści się w podanym przedziale <min;max>

\item Zaproponowana funkcja \textbf{zwiekz\_pensje}, która na podstawie id pracownika zmienia wartość kolumny pensja zgodnie z zasadami:
\begin{itemize}
\item jeżeli pracownik nie ma zwierzchnika jego pensja nie ulega zmianie
\item jeżeli pracownik ma zwierzchnika, ale ma prowizje, pensja zwiększa się o wartość 100
\item jeżeli pracownik ma zwierzchnika, ale nie ma prowizji, pensja zwiększa się o 10\%
\item funkcja zwraca nową pensję jako wartość
\end{itemize}

\item Wywołanie komendy usunięcia dowolnej procedury i funkcji

\item Zaproponowany typ pozwalający przechowywać dane obiektu \textbf{ETAT} oraz tabelę etatów

\item Zaproponowana funkcja zwracająca tabelę z danymi obiektów \textbf{ETAT} zaproponowanymi w poprzedznim zadaniu. Dane(id, etat):(`100',`stażysta'),(`101',`junior'),(`102',`medium'),(`103',`master')

\item Zaproponowana funkcja \textbf{zwroc\_etaty}, która zwraca wszystkie etaty w formie tabeli
\end{enumerate}

\section{Wnioski}


\end{document}
