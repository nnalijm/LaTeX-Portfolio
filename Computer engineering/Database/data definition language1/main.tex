\documentclass{article}

\usepackage{../packages}

\graphicspath{{./figures}}


\begin{document}
\begin{titlepage}
\begin{center}
	\includegraphics[scale=0.7]{logo.png}

	\vspace*{4cm}
	\textbf{Bazy danych\\ Laboratorium}

	\vspace{1.5cm}
	\textit{Zapytania DDL SQL Oracle 1}

	\vspace{1.5cm}
	\textbf{Stanislau Antanovich}\\
	nr. indeksu: 173590\\
	gr. lab: L04

	\vspace{4.5cm}
\end{center}
\end{titlepage}

\tableofcontents
\listoffigures
\lstlistoflistings

\newpage
 
\section{Wprowadzenie}

\subsection{Cel ćwiczenia}

\blindtext
\subsection{Przygotowanie}

\blindtext

\section{Realizacja}

\begin{enumerate}
\item Tworzenie tabeli \emph{TEST} zawierającej 4 przykładowe pola przechowujące informacje z Identyfikatorem, tekstem, czasem i wartością zmiennoprzecinkową

\begin{lstlisting}[style=SQL, caption=\textit{Tworzenie tabeli ``TEST'' zawirającej 4 przykładowe pola przechowujące informacje z Identyfikatorem, tekstem, czasem i warością zmiennoprzecinkową}]
CREATE TABLE TEST (ID INT PRIMARY KEY, TEKST VARCHAR(255), CZAS TIMESTAMP, WARTOSC FLOAT);
\end{lstlisting}

\begin{figure}[H]
	\centering
	\includegraphics[scale=1]{zadanie1.png}
	\caption{\textit{Tworzenie tabeli ``TEST'' zawirającej 4 przykładowe pola przechowujące informacje z Identyfikatorem, tekstem, czasem i warością zmiennoprzecinkową}}
\end{figure}

\item Usunięcie stworzonej tabeli odpowiednią komendą SQL.

\begin{lstlisting}[style=SQL,caption=\textit{Usunięcie stworzonej tabeli odpowiednią komendą SQL}]
DROP TABLE TEST;
\end{lstlisting}

\begin{figure}[H]
	\centering
	\includegraphics[scale=1]{zadanie2.png}
	\caption{\textit{Usunięcie stworzonej tabeli odpowiednią komendą SQL}}
\end{figure}

\item Tworzenie tabeli \emph{ZADANIA(TASKS)} zawierającą następujące kolumny:
\begin{itemize}
\item ID\_ZADANIA
\item NAZWA
\item DATA\_ROZPOCZECIA
\item DATA\_ZAKONCZENIA
\end{itemize}

\begin{lstlisting}[style=SQL,caption=\textit{Tworzenie tabeli ``ZADANIA''}]
CREATE TABLE ZADANIA (
	ID_ZADANIA INT PRIMARY KEY,
	NAZWA VARCHAR(255),
	DATA_ROZPOCZECIA DATE,
	DATA_ZAKONCZENIA DATE
);
\end{lstlisting}

\begin{figure}[H]
	\centering
	\includegraphics[scale=1]{zadanie3.png}
	\caption{\textit{Tworzenie tabeli ``ZADANIA''}}
\end{figure}

\item Dodanie nowej kolumny \emph{KOMENTARZ}(ALTER)

\begin{lstlisting}[style=SQL,caption=\textit{Dodanie nowej kolumny ``KOMENTARZ''}]
ALTER TABLE ZADANIA ADD KOMENTARZ VARCHAR(255);
\end{lstlisting}

\begin{figure}[H]
	\centering
	\includegraphics[scale=1]{zadanie4.png}
	\caption{\textit{Dodanie nowej kolumny ``KOMENTARZ''}}
\end{figure}

\item Realizacja powiązania tabeli \emph{ZADANIA} z tabelą \emph{PRACOWNICY}

\begin{lstlisting}[style=SQL,caption=\textit{Realizacja powiązania tabeli ``ZADANIA'' z tabelą ``PRACOWNICY''}]
ALTER TABLE ZADANIA ADD ID_PRACOWNIKA NUMBER(4);
ALTER TABLE ZADANIA ADD CONSTRAINT FK_PRACOWNIK FOREIGN KEY (ID_PRACOWNIKA) REFERENCES PRACOWNICY(ID_PRACONIKA);
\end{lstlisting}

\begin{figure}[H]
	\centering
	\includegraphics[scale=1.2]{zadanie5.png}
	\caption{\textit{Realizacja powiązania tabeli ``ZADANIA'' z tabelą ``PRACOWNICY''}}
\end{figure}

\item Modyfikacja tabeli \emph{ZADANIA} w taki sposób, aby rekordy miały określone ograniczenia:
\begin{itemize}
\item ID\_ZADANIE -- PRIMARY KEY
\item NAZWA -- UNIQUE
\item DATA\_ROZPOCZECIA -- NOT NULL
\item KOMENTARZ -- NOT NULL, DEFAULT
\item ID\_PRACOWNIKA -- FOREIGN KEY
\end{itemize}

\begin{lstlisting}[style=SQL, caption=\textit{Modyfikacja tabeli ``ZADANIA''}]
CREATE TABLE ZADANIA (
    ID_ZADANIA INT PRIMARY KEY,
    NAZWA VARCHAR(100) UNIQUE,
    DATA_ROZPOCZECIA DATE NOT NULL,
    DATA_ZAKONCZENIA DATE,
    KOMENTARZ VARCHAR(255) NOT NULL DEFAULT 'KOMENTARZ',
    ID_PRACOWNIKA INT,
    FOREIGN KEY (ID_PRACOWNIKA) REFERENCES PRACOWNICY(ID_PRACOWNIKA)
);
\end{lstlisting}

\begin{figure}[H]
	\centering
	\includegraphics[scale=1.2]{zadanie6.png}
	\caption{\textit{Modyfikacja tabeli ``ZADANIA''}}
\end{figure}

\item Dodanie do tabeli \emph{ZADANIA} siedem przykładowych zadań powiązanych odpowiednią relacją z tabelą \emph{PRACOWNICY}. Jeden pracownik musi posiadać przynajmniej trzy zadania.

\begin{lstlisting}[style=SQL, caption=\textit{Dodanie do tabeli ``ZADANIA''}]
INSERT INTO ZADANIA (ID_ZADANIE, NAZWA, DATA_ROZPOCZECIA, DATA_ZAKONCZENIA, KOMENTARZ, ID_PRACOWNIKA) VALUES (1, 'ZADANIE 1', '2024-05-12', '2024-05-13', 'ZADANIE 1', 7369);

INSERT INTO ZADANIA (ID_ZADANIE, NAZWA, DATA_ROZPOCZECIA, DATA_ZAKONCZENIA, KOMENTARZ, ID_PRACOWNIKA) VALUES (2, 'ZADANIE 2', '2024-05-13', '2024-05-14', 'ZADANIE 2', 7369);

INSERT INTO ZADANIA (ID_ZADANIE, NAZWA, DATA_ROZPOCZECIA, DATA_ZAKONCZENIA, KOMENTARZ, ID_PRACOWNIKA) VALUES (3, 'ZADANIE 3', '2024-05-14', '2024-05-15', 'ZADANIE 3', 7369);

INSERT INTO ZADANIA (ID_ZADANIE, NAZWA, DATA_ROZPOCZECIA, DATA_ZAKONCZENIA, KOMENTARZ, ID_PRACOWNIKA) VALUES (4, 'ZADANIE 4', '2024-05-15', '2024-05-16', 'ZADANIE 4', 7499);

INSERT INTO ZADANIA (ID_ZADANIE, NAZWA, DATA_ROZPOCZECIA, DATA_ZAKONCZENIA, KOMENTARZ, ID_PRACOWNIKA) VALUES (5, 'ZADANIE 5', '2024-05-16', '2024-05-17', 'ZADANIE 5', 7499);

INSERT INTO ZADANIA (ID_ZADANIE, NAZWA, DATA_ROZPOCZECIA, DATA_ZAKONCZENIA, KOMENTARZ, ID_PRACOWNIKA) VALUES (6, 'ZADANIE 6', '2024-05-17', '2024-05-18', 'ZADANIE 6', 7505);

INSERT INTO ZADANIA (ID_ZADANIE, NAZWA, DATA_ROZPOCZECIA, DATA_ZAKONCZENIA, KOMENTARZ, ID_PRACOWNIKA) VALUES (7, 'ZADANIE 7', '2024-05-18', '2024-05-19', 'ZADANIE 7', 7505);
\end{lstlisting}

\begin{figure}[H]
	\centering
	\includegraphics[scale=1.2]{zadanie7.png}
	\caption{\textit{Dodanie do tabeli ``ZADANIA''}}
\end{figure}

\item Wyświetlenie następującego wyniku z użyciem zapytania SQL(nagłówek tabeli wynikowej): ID\_PRACOWNIKA| NAZWISKO|IMIE|ILOSC ZADAN

\begin{lstlisting}[style=SQL, caption=\textit{Wyświetlenie następującego wyniku z użyciem zapytania SQL}]
SELECT P.ID_PRACOWNIKA, P.NAZWISKO, P.IMIE, COUNT(Z.ID_ZADANIE) AS "ILOSC ZADAN" FROM PRACOWNICY P LEFT JOIN ZADANIA Z ON P.ID_PRACOWNIKA = Z.ID_PRACOWNIKA GROUP BY P.ID_PRACOWNIKA, P.NAZWISKO, P.IMIE;
\end{lstlisting}

\begin{figure}[H]
	\centering
	\includegraphics[scale=1]{zadanie8.png}
	\caption{\textit{Wyświetlenie następującego wyniku z użyciem zapytania SQL}}
\end{figure}

\item Wyświetlenie następującego wyniku z użyciem zapytania SQL(nagłówek tabeli wynikowej): ID\_PRACOWNIKA| NAZWISKO|IMIE|ILOSC TRWAJACYCH ZADAN

\begin{lstlisting}[style=SQL, caption=\textit{Wyświetlenie następującego wyniku z użyciem zapytania SQL}]
SELECT P.ID_PRACOWNIKA, P.NAZWISKO, P.IMIE, COUNT(Z.ID_ZADANIE) AS "ILOSC TRWAJACYCH ZADAN" FROM PRACOWNICY P INNER JOIN ZADANIA Z ON P.ID_PRACOWNIKA = Z.ID_PRACOWNIKA WHERE Z.DATA_ZAKONCZENIA >= SYSDATE GROUP BY P.ID_PRACOWNIKA, P.NAZWISKO, P.IMIE;
\end{lstlisting}

\begin{figure}[H]
	\centering
	\includegraphics[scale=1.2]{zadanie9.png}
	\caption{\textit{Wyświetlenie następującego wyniku z użyciem zapytania SQL}}
\end{figure}

\item Wyświetlenie wszystkich pracowników posiadających przynajmniej dwa zadania

\begin{lstlisting}[style=SQL, caption=\textit{Wyświetlenie wszystkich pracowników posiadających przynajmniej dwa zadania}]
SELECT P.ID_PRACOWNIKA, P.NAZWISKO, P.IMIE FROM PRACOWNICY P INNER JOIN ZADANIA Z ON P.ID_PRACOWNIKA = Z.ID_PRACOWNIKA GROUP BY P.ID_PRACOWNIKA, P.NAZWISKO, P.IMIE HAVING COUNT(Z.ID_ZADANIE) >= 2;
\end{lstlisting}

\begin{figure}[H]
	\centering
	\includegraphics[scale=1.2]{zadanie10.png}
	\caption{\textit{Wyświetlenie wszystkich pracowników posiadających przynajmniej dwa zadania}}
\end{figure}


\end{enumerate}
\section{Wnioski}

Po tych zadaniach zdobyłem solidne podstawy w projektowaniu i zarządzaniu bazami danych. Teraz czuję się pewniej w tworzeniu tabel, definiowaniu ograniczeń, manipulowaniu danymi oraz optymalizacji zapytań SQL. Te umiejętności będą nieocenione w mojej pracy, umożliwiając efektywne zarządzanie danymi i analizę informacji w różnych kontekstach.

\end{document}
