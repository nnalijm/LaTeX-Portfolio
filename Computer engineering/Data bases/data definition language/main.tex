\documentclass{article}

\usepackage{graphicx}
\usepackage[T1]{fontenc}
\usepackage[polish]{babel}
\usepackage[utf8]{inputenc}
\usepackage{listings}
\usepackage{blindtext}


\graphicspath{{./figures}}

\renewcommand\contentsname{Spis treści}
\renewcommand\listfigurename{Spis rysunków}

\begin{document}
\begin{titlepage}
\begin{center}
	\includegraphics[scale=0.7]{logo.png}

	\vspace*{4cm}
	\textbf{Bazy danych\\ Laboratorium}

	\vspace{1.5cm}
	\textit{Zapytania DDL SQL Oracle}

	\vspace{1.5cm}
	\textbf{Stanislau Antanovich}\\
	nr. indeksu: 173590\\
	gr. lab: L04

	\vspace{4.5cm}
	\today
\end{center}
\end{titlepage}

\tableofcontents
\listoffigures
\lstlistoflistings

\newpage
 
\section{Wprowadzenie}

\subsection{Cel ćwiczenia}

\blindtext
\subsection{Przygotowanie}

\blindtext

\section{Realizacja}

\begin{enumerate}
\item Tworzenie tabeli \emph{TEST} zawierającą 4 przykładowe pola przechowujące informacje z Identyfikatorem, tekstem, czasem i wartością zmiennoprzecinkową

\item Usunięcie stworzonej tabeli odpowiednią komendą SQL.

\item Tworzenie tabeli \emph{ZADANIA(TASKS)} zawierającą następujące kolumny:
\begin{itemize}
\item ID\_ZADANIE
\item NAZWA
\item DATA\_ROZPOCZECIA
\item DATA\_ZAKONCZENIA
\end{itemize}

\item Dodanie nowej kolumny \emph{KOMENTARZ}(ALTER)

\item Realizacja powiązania tabeli \emph{ZADANIA} z tabelą \emph{PRACOWNICY}

\item Modyfikacja tabeli \emph{ZADANIA} w taki sposób, aby rekordy miały określone ograniczenia:
\begin{itemize}
\item ID\_ZADANIE -- PRIMARY KEY
\item NAZWA -- UNIQUE
\item DATA\_ROZPOCZECIA -- NOT NULL
\item KOMENTARZ -- NOT NULL, DEFAULT
\item ID\_PRACOWNIKA -- FOREIGN KEY
\end{itemize}

\item Dodanie do tabeli \emph{ZADANIA} siedem przykładowych zadań powiązanych odpowiednią relacją z tabelą \emph{PRACOWNICY}. Jeden pracownik musi posiadać przynajmniej trzy zadania.

\item Wyświetlenie następującego wyniku z użyciem zapytania SQL(nagłówek tabeli wynikowej): ID\_PRACOWNIKA|NAZWISKO|IMIE|ILOSC ZADAN

\item Wyświetlenie następującego wyniku z użyciem zapytania SQL(nagłówek tabeli wynikowej): ID\_PRACOWNIKA|NAZWISKO|IMIE|ILOSC TRWAJACYCH ZADAN

\item Wyświetlenie wszystkich pracowników posiadających przynajmniej dwa zadania

\end{enumerate}
\section{Wnioski}

\end{document}
