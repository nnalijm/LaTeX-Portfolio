\documentclass[a4paper, 10pt]{article}

\usepackage{graphicx}
\usepackage[T1]{fontenc}
\usepackage[polish]{babel}
\usepackage[utf8]{inputenc}
\usepackage{listings}
\usepackage{xcolor}
\usepackage{color}


\graphicspath{{./figures}}

\renewcommand\contentsname{Spis treści}
\renewcommand\listfigurename{Spis rysunków}
\renewcommand\lstlistingname{Polecenie}
\renewcommand\lstlistlistingname{Spis poleceń}

\definecolor{codegreen}{rgb}{0,0.6,0}
\definecolor{codegray}{rgb}{0.5,0.5,0.5}
\definecolor{codepurple}{HTML}{C42043}
\definecolor{backcolour}{HTML}{F2F2F2}
\definecolor{bookColor}{cmyk}{0,0,0,0.90}
\color{bookColor}

\lstdefinestyle{SQL}{
    language=SQL,
    backgroundcolor=\color{backcolour},
    basicstyle=\footnotesize\ttfamily,
    breaklines=true,
    captionpos=b,
    commentstyle=\color{green},
    keywordstyle=\color{red},
    stringstyle=\color{red},
    showstringspaces=false,
    tabsize=2,
    morekeywords={USE, CREATE, TABLE, VARCHAR, INT, NOT, NULL, PRIMARY, KEY, AUTO_INCREMENT, INSERT, INTO, VALUES, SELECT, FROM, WHERE, ORDER, BY, ASC, DESC, GROUP, HAVING, UPDATE, SET, DELETE, JOIN, LEFT, RIGHT, OUTER, INNER, ON},
    frame=single,
    framesep=5pt,
    frameround=tttt,
    rulecolor=\color{gray}
}

\begin{document}
\begin{titlepage}
\begin{center}
	\includegraphics[scale=0.7]{logo.png}

	\vspace*{4cm}
	\textbf{Bazy danych\\ Laboratorium}

	\vspace{1.5cm}
	\textit{Wprowadzenie do Oracle}

	\vspace{1.5cm}
	\textbf{Stanislau Antanovich}\\
	nr. indeksu: 173590\\
	gr. lab: L04

	\vspace{4.5cm}
	\today
\end{center}
\end{titlepage}

\tableofcontents
\listoffigures
\lstlistoflistings

\newpage

\section{Wprowadzenie}
\subsection{Cel ćwiczenia}

Celem tego laboratorium jest zapoznanie się z narzędziem Oracle SQL Developer oraz praktyczne zastosowanie wiedzy na temat tworzenia, zarządzania i manipulowania bazami danych w Oracle. 

Poprzez realizację konkretnych zadań na przykładzie bazy danych ``Firma handlowa'', jest możliwość zdobycia umiejętności w obszarze tworzenia zapytań SQL, importowania i eksportowania danych, jak również zarządzania nimi przy użyciu interfejsu SQL Developer.

\subsection{Przygotowanie}

\begin{enumerate}
\item Zapoznanie się z narzędziem Oracle SQL Developer.
\item Przeanalizowanie struktury(tabele, pola, typy pól) pobranej bazy danych.
\item Zaimportowanie bazy danych ``Firma handlowa'' do narzędzia SQL Developer.
\end{enumerate}

\section{Realizacja}

Po zaimportowaniu bazy danych ``Firma handlowa'' do narzędzia SQL developer można zaczynać wykonywać polecenia SQL.

\begin{enumerate}
\item Wyświetlienie wszystkiej informacji o pracownikach 
\item Wyświetlienie informacji o imieniu, nazwisku i pensji pracowników
\item Wypisywanie wszystkich Pracowników, sortując na podstawie Nazwiska w kolejności przeciwnej do alfabetycznej
\item Wyświetlienie informacji o imieniu, nazwisku i pensji pracowników, których pensja jest > 1500
\item Wyświetlienie zamówień o wartości z przedziału <1000,3000>, złożonych po dniu 10-05-1991
\item Sortowanie wyniku zadania 5 po datach złożenia zamówień i po wartościach zamówień
\item Dodawanie pracownika o danych z przykładowymi wartościami
\item Modyfikacja płacę dodanego pracownika do wartości 2000. 
\item Usunięcie dodanego pracownika w podpunkcie. 
\item Wyświetlienie imia i nazwiska i nazwę etatu pracowników zatrudnionych na etacie  ANALYST. Wybieranie danych z dwóch tabel, należy zastosować złączenie (odpowiedni warunek w WHERE lub klauzulę JOIN). 
\item Wyświetlienie wszystkich pracowników na etacie MANAGER, wynik posortowany po identyfikatorze.
\end{enumerate}


\end{document}
