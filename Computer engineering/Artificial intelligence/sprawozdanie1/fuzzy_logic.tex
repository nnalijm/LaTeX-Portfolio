\documentclass[a4paper, 10pt]{article}

% PACKAGES
\usepackage[T1]{fontenc}
\usepackage[polish]{babel}
\usepackage[utf8]{inputenc}
\usepackage{graphicx}
\usepackage{listings}
\usepackage{xcolor}
\usepackage{caption}
\usepackage{float}
%\usepackage[left=25mm, right=25mm, top=45mm, bottom=45mm]{geometry}
\usepackage{hyperref}
\usepackage{import}

% PREAMBLE
% folder which contains images
\graphicspath{{./figures/}}

\renewcommand\contentsname{Spis treści}
\renewcommand\listfigurename{Spis rysunków}
\lstset{language=Python,                     
	basicstyle=\scriptsize, 
	breakatwhitespace=false,
	breaklines=true,
	commentstyle=\color{green}\ttfamily,
	frame=single,
	keepspaces=false,
	keywordstyle=\color{blue}\ttfamily,
	numbers=left,
	numbersep=5pt,
	showspaces=false,
	showstringspaces=false,
	showtabs=false,	
	stringstyle=\color{magenta}\ttfamily,
	tabsize=2,
	morecomment=[l][\color{red}]{\#}
}

\hypersetup{
	colorlinks=true,
	linkcolor=blue,
	filecolor=magenta,
	urlcolor=cyan,
	pdftitle={Synteza układu wnioskującego},
	pdfpagemode=FullScreen,
}

\begin{document}

\begin{titlepage}
\begin{center}
	\includegraphics[scale=0.7]{logo.png}

	\vspace*{4cm}
	\textbf{Sztuczna inteligencja\\ Laboratorium}

	\vspace{1.5cm}
	\textit{Synteza układu wnioskującego}

	\vspace{1.5cm}
	\textbf{Stanislau Antanovich}\\
	nr. indeksu: 173590\\
	gr. lab: L04
\end{center}
\end{titlepage}

\tableofcontents
\listoffigures

\newpage

\section{Wstęp}\label{sec:wstep}
\subsection{Cel ćwiczenia}\label{subsec:cel}

Celem ćwiczenia jest zapoznanie się ze sposobem syntezy systemu ekspertowego pełniącego rolę regulatora. Rozmyty system ekspertowy zostanie użyty do sterowania małym robotem mobilnym Khepera realizującym zadanie omijania przeszkód. 

Laboratorium składa się z trzech zasadniczych części. Część I ma na celu zapoznanie się z budową oraz ze sposobem sterowania robotem mobilnym Khepera III za pomocą dedykowanych funkcji języka Python. W części II zostanie zaprojektowany system ekspertowego typu Mamdaniego z wykorzystaniem biblioteki scikit-learn realizujący zadanie omijania przeszkód. Część III laboratorium polega na praktycznej weryfikacji poprawności zrealizowanego systemu w sterowaniu rzeczywistym robotem.

\import{./sections/}{introduction.tex}

\section{System ekspertowy typu Mamdaniego}\label{sec:c1}
\subsection{Problem}

Zaprojektować rozmyty układ ekspertowy doradzający ile napiwku pozostawić w restauracji na podstawie oceny jakości obsługi oraz jakości jedzenia. Jakość obsługi i jakość jedzenia będzie oceniana w skali od 1 do 10, gdzie 10 reprezentuję ocenę maksymalną, natomiast napiwek będzie liczbą z przedziału [0,30] reprezentującą procent wartości rachunku.

Baza reguł będzie składała się z 5 reguł. System zostanie wykonany w dwóch etapach. W \textbf{etapie pierwszym}(rys. \ref{fig:obsluga} i \ref{fig:napiwek}) system będzie zbudowany z jednego wejścia(\emph{obsługa}) i jednego wyjścia(\emph{napiwek}) oraz 3 reguł postaci:
\begin{itemize}
	\item[R1] \emph{jeżeli obsługa jest słaba, to napiwek jest mały}
	\item[R2] \emph{jeżeli obsługa jest dobra, to napiwek jest średni}
	\item[R3] \emph{jeżeli obsługa jest wspaniała, to napiwek jest duży}
\end{itemize}

\begin{figure}[H]
	\centering
	\includegraphics[scale=0.25]{Figure_1.png}
	\caption{\textit{Obsługa}}
	\label{fig:obsluga}
\end{figure}

\begin{figure}[H]
	\centering
	\includegraphics[scale=0.25]{Figure_2.png}
	\caption{\textit{Napiwek}}
	\label{fig:napiwek}
\end{figure}

W \textbf{etapie drugim} do systemu zostanie dodane gruga zmienna wejściowa \emph{jedzenie}(rys. \ref{fig:jedzenie}) oraz 2 dodatkowe reguły
\begin{itemize}
	\item[R4] \emph{jeżeli jedzenie jest zepsute, to napiwek jest mały}
	\item[R5] \emph{jeżeli jedzenie jest wyborne, to napiwek jest duży}
\end{itemize}


\subsection{Realizacja}
\subsubsection{Inicjowanie modułów}\label{p1}

W tej sekcji inicjujemy wszystkie wymagane biblioteki dla prawidłowego działania programu.

\lstinputlisting[firstline=0, lastline=4]{./code/main.py}

\subsubsection{Tworzenie zmiennych stanu poprzednika ``obsługa'' oraz następnika ``napiwek''}\label{p2}

W tej sekcji tworzymy zmienne stanu poprzednika ``obsługa'' oraz następnika ``napiwek''.

\lstinputlisting[firstline=6, lastline=7]{./code/main.py}

\subsubsection{Dodanie zbiorów rozmytych}\label{p3}
 
W tej sekcji do zminnej \emph{obsługa} dodajemy następujące zbiory: \emph{slaba, dobra, wspaniala}. 

Zbiór \emph{slaba} o centrum umieszczonym w punkcie uniwersum równym \textbf{0} i rozpiętości wynoszącej \textbf{1.5}. Dla zbiorów \emph{dobra} i \emph{wspaniala} o centrach ulokowanych w punktach odpowiednio \textbf{5} oraz \textbf{10} i rozpiętości wynoszącej \textbf{1.5}

Dla zmiennej \emph{napiwek} dodajemy zbiory trójkątne: \emph{maly, sredni, duzy} o parametrach [0, 5, 10], [10, 15, 20] i [20, 25, 30] odpowiednio.

\lstinputlisting[firstline=9, lastline=15]{./code/main.py}

\subsubsection{Podgląd zbiorów rozmytych}\label{p4}

Podgląd zbiorów rozmytch można zrealizować metodą \verb|view()| dla poszczególnych zmiennych stanu.

\lstinputlisting[firstline=17, lastline=18]{./code/main.py}

\subsubsection{Definicja reguł}\label{p5}

W tej sekcji definiujemy reguły.

\lstinputlisting[firstline=20, lastline=22]{./code/main.py}

\subsubsection{Dodanie reguł do systemu rozmytego}\label{p6}

W tej sekcji dodajemy powyżej zdefiniowane reguły do systemu rozmytego.

\lstinputlisting[firstline=24, lastline=25]{./code/main.py}

\subsubsection{Sprawdzenie działania systemu}\label{p7}

W tej sekcji sprawdzamy działanie systemu dla wartości obsługi równej \textbf{0}(rys. \ref{fig:obsluga0}) oraz dla wartości równej \textbf{10}(rys. \ref{fig:obsluga10}).

\lstinputlisting[firstline=27,lastline=31]{./code/main.py}

Wartość napiwku dla wartości obsługi wynosi: 5.07657801

Wartość napiwku dla wartości obsługi wynosi: 24.9234219

\begin{figure}[H]
	\centering
	\includegraphics[scale=0.25]{Figure_3.png}
	\caption{\textit{Wyostrzenie metodą środka ciężkości dla wejścia obsługa = 0}}
	\label{fig:obsluga0}
\end{figure}

\begin{figure}[H]
	\centering
	\includegraphics[scale=0.25]{Figure_4.png}
	\caption{\textit{Wyostrzenie metodą środka ciężkości dla wejścia obsługa = 10}}
	\label{fig:obsluga10}

\end{figure}

\subsubsection{Sprawdzenie działania systemu dla wartości obsługi}\label{p8}

W tej sekcji sprawdzamy działanie systemu dla wartości obsługi od 0 do 10(rys. \ref{fig:napiwek_obsluga} wykres funkcji napiwek=f(obsluga)).

\lstinputlisting[firstline=33, lastline=45]{./code/main.py}

\begin{figure}[H]
	\centering
	\includegraphics[scale=0.25]{Figure_5.png}
	\caption{\textit{Powierzchnia prejścia systemu ``napiwek'' o jednym wejściu}}
	\label{fig:napiwek_obsluga}
\end{figure}

\subsubsection{Dodanie drugiej wejściowej}\label{p9}
 
W tej selcji dodajemy drugą wejściową zmienną stanu ``jedzenie'' o trapezoidalnych(\verb|trapmf|) zbiorach rozmytych ``zepsute'' oraz ``wyborne''.

\lstinputlisting[firstline=17, lastline=19]{./code/main.py}

\subsubsection{Dodanie reguł 4 i 5}\label{p10}

W tej sekcji dodajemy reguły 4 i 5 analogicznie do punktu \ref{p5}.
\subsubsection{Sprawdzenie działania systemu dla wartości ``obsługi''}\label{p11}

W tej sekcji sprawdzamy działanie systemu dla wartości obsługi równej \textbf{0} oraz wartości jedzenia równej \textbf{0}.

\subsubsection{Sprawdzenie działania systemu dla wartości ``obsługi'' i ``jedzenia''}\label{p12}

W tej sekcji sprawdzamy działanie systemu dla wartości obsługi i jedzenia od 0 do 10(rys. \ref{fig:napiwek_obsluga_jedzenie})

\begin{figure}[H]
	\centering
	\includegraphics[scale=0.25]{Figure_6.png}
	\caption{\textit{Powierzchnia przejścia systemu ``napiwek'' o dwóch wejściach}}
	\label{fig:napiwek_obsluga_jedzenie}
\end{figure}


\section{Modyfikacja systemu}\label{sec:c2}

\section{Przykładowe zadania zaliczeniowe}\label{sec:c3}
 
\section{Wnioski}

\end{document}
