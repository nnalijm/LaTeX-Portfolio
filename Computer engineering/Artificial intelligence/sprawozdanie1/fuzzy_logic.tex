\documentclass[a4paper, 10pt]{article}

% PACKAGES
\usepackage[T1]{fontenc}
\usepackage[polish]{babel}
\usepackage[utf8]{inputenc}
\usepackage{graphicx}
\usepackage{listings}
\usepackage{xcolor}
\usepackage{caption}
\usepackage{float}
%\usepackage[left=25mm, right=25mm, top=45mm, bottom=45mm]{geometry}
\usepackage{hyperref}

% PREAMBLE
% folder which contains images
\graphicspath{{./pix/}}

\renewcommand\contentsname{Spis treści}
\renewcommand\listfigurename{Spis rysunków}
\lstset{language=Python,                     
	basicstyle=\scriptsize, 
	breakatwhitespace=false,
	breaklines=true,
	commentstyle=\color{green}\ttfamily,
	frame=single,
	keepspaces=false,
	keywordstyle=\color{blue}\ttfamily,
	numbers=left,
	numbersep=5pt,
	showspaces=false,
	showstringspaces=false,
	showtabs=false,	
	stringstyle=\color{red}\ttfamily,
	tabsize=2,
	morecomment=[l][\color{magenta}]{\#}
}

\hypersetup{
	colorlinks=true,
	linkcolor=blue,
	filecolor=magenta,
	urlcolor=cyan,
	pdftitle={Synteza układu wnioskującego},
	pdfpagemode=FullScreen,
}

\begin{document}

\begin{titlepage}
\begin{center}
	\includegraphics[scale=0.7]{logo.png}

	\vspace*{4cm}
	\textbf{Sztuczna inteligencja\\ Laboratorium}

	\vspace{1.5cm}
	\textit{Synteza układu wnioskującego}

	\vspace{1.5cm}
	\textbf{Stanislau Antanovich}\\
	nr. indeksu: 173590\\
	gr. lab: L04
\end{center}
\end{titlepage}

\tableofcontents
\listoffigures

\newpage
\section{Wstęp}
\subsection{Cel ćwiczenia}

Laboratorium składa się z trzech zasadniczych części. Część \ref{c1} ma na celu zapoznanie się ze sposobem syntezy rozmytego systemu ekspertowego typu Mamdaniego z wykorzystaniem biblioteki \textbf{scikit-fuzzy}. W części \ref{c2}, należy zapoznać się z ideą działania systemu Mamdaniego a następnie dokonać modyfikacji systemu wykonanego w części \ref{c1}. Część \ref{c3} laboratorium polega na wykonaniu przykładowego zadania zaliczeniowego.

\section{System ekspertowy typu Mamdaniego}\label{c1}
\subsection{Problem}

Zaprojektować rozmyty układ ekspertowy doradzający ile napiwku pozostawić w restauracji na podstawie oceny jakości obsługi oraz jakości jedzenia. Jakość obsługi i jakość jedzenia będzie oceniana w skali od 1 do 10, gdzie 10 reprezentuję ocenę maksymalną, natomiast napiwek będzie liczbą z przedziału [0,30] reprezentującą procent wartości rachunku.

Baza reguł będzie składała się z 5 reguł. System zostanie wykonany w dwóch etapach. W \textbf{etapie pierwszym}(rys. \ref{fig:obsluga} i \ref{fig:napiwek}) system będzie zbudowany z jednego wejścia(\emph{obsługa}) i jednego wyjścia(\emph{napiwek}) oraz 3 reguł postaci:
\begin{itemize}
	\item[R1] \emph{jeżeli obsługa jest słaba, to napiwek jest mały}
	\item[R2] \emph{jeżeli obsługa jest dobra, to napiwek jest średni}
	\item[R3] \emph{jeżeli obsługa jest wspaniała, to napiwek jest duży}
\end{itemize}

\begin{figure}[H]
	\centering
	\includegraphics[scale=0.25]{Figure_1.png}
	\caption{\textit{Obsługa}}
	\label{fig:obsluga}
\end{figure}

\begin{figure}[H]
	\centering
	\includegraphics[scale=0.25]{Figure_2.png}
	\caption{\textit{Napiwek}}
	\label{fig:napiwek}
\end{figure}

W \textbf{etapie drugim} do systemu zostanie dodane gruga zmienna wejściowa \emph{jedzenie}(rys. \ref{fig:jedzenie}) oraz 2 dodatkowe reguły
\begin{itemize}
	\item[R4] \emph{jeżeli jedzenie jest zepsute, to napiwek jest mały}
	\item[R5] \emph{jeżeli jedzenie jest wyborne, to napiwek jest duży}
\end{itemize}


\subsection{Realizacja}
\subsubsection{Inicjowanie modułów}

W tej sekcji inicjujemy wszystkie wymagane biblioteki dla prawidłowego działania programu.

\lstinputlisting[firstline=0, lastline=4]{./code/main.py}

\subsubsection{Tworzenie zmiennych stanu poprzednika ``obsługa'' oraz następnika ``napiwek''}

W tej sekcji tworzymy zmienne stanu poprzednika ``obsługa'' oraz następnika ``napiwek''.

\lstinputlisting[firstline=6, lastline=7]{./code/main.py}

\subsubsection{Dodanie zbiorów rozmytych}
 
W tej sekcji do zminnej \emph{obsługa} dodajemy następujące zbiory: \emph{slaba, dobra, wspaniala}. 

Zbiór \emph{slaba} o centrum umieszczonym w punkcie uniwersum równym \textbf{0} i rozpiętości wynoszącej \textbf{1.5}. Dla zbiorów \emph{dobra} i \emph{wspaniala} o centrach ulokowanych w punktach odpowiednio \textbf{5} oraz \textbf{10} i rozpiętości wynoszącej \textbf{1.5}

Dla zmiennej \emph{napiwek} dodajemy zbiory trójkątne: \emph{maly, sredni, duzy} o parametrach [0, 5, 10], [10, 15, 20] i [20, 25, 30] odpowiednio.

\lstinputlisting[firstline=9, lastline=15]{./code/main.py}

\subsubsection{Podgląd zbiorów rozmytych}
\subsubsection{Definicja reguł}
\subsubsection{Dodanie reguł do systemu rozmytego}
\subsubsection{Sprawdzenie działania systemu}
\subsubsection{Sprawdzenie działania systemu dla wartości obsługi}
\subsubsection{Dodanie drugiej wejściowej}
\subsubsection{Dodanie reguł 4 i 5}
\subsubsection{Sprawdzenie działania systemu dla wartości ``obsługi''}
\subsubsection{Sprawdzenie działania systemu dla wartości ``obsługi'' i ``jedzenia''}


\section{Modyfikacja systemu}\label{c2}

\section{Przykładowe zadania zaliczeniowe}\label{c3}

\section{Wnioski}

\end{document}
