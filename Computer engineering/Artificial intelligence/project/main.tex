\documentclass[a4paper, 10pt]{article}

\usepackage{../packages}

\graphicspath{{./figures}}

\hypersetup{                                                                                          
	colorlinks=true,                                                                              
	linkcolor=blue,                                                                               
	filecolor=magenta,                                                                            
	urlcolor=cyan,                                                                                
	pdftitle={Sztuczna Inteligencja},                                                            
	pdfpagemode=FullScreen,
}

\begin{document}
\begin{titlepage}
\begin{center}
	\includegraphics[scale=0.7]{logo.png}

	\vspace*{4cm}
	\textbf{Sztuczna inteligencja\\ Projekt}

	\vspace{1.5cm}
	\textit{Spambase\\MLP}

	\vspace{1.5cm}
	\textbf{Stanislau Antanovich}\\
	173590\\

	\vspace{6.5cm}
	Rzeszów, 2024
\end{center}
\end{titlepage}

% contents
\tableofcontents

% figures
\listoffigures

% commands
%\lstlistoflistings

\newpage

\section{Opis projektu}
\subsection{Cel projektu}

Celem projektu jest zastosowanie sieci neuronowej typu MLP(Multilayer perceptron)
\subsection{Opis i przygotowanie wykorzystanych danych}

\section{Część teoretyczna}
\subsection{Algorytm MLP}

Perceptron wielowartwowy(MLP) to typ sztucznych sieci neuronowych. Sieć tego typu składa się z wielu warstw neuronów połączonych kaskadowo(stąd nazwa ``wielowarstwowy'').

\section{Skrypt programu}

\section{Eksperymenty}
\subsection{Eksperyment 1}
\subsection{Eksperyment 2}
\subsection{Eksperyment 3}

\section{Podsumowania i wnioski}

\end{document}
