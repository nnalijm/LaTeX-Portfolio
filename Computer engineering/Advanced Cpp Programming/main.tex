\documentclass[a4paper, 10pt]{article}

\usepackage{float}
\usepackage{graphicx}
\usepackage{listings}
\usepackage[polish]{babel}
\usepackage[utf8]{inputenc}
\usepackage[T1]{fontenc}
\usepackage{enumitem}
\usepackage{caption}
\usepackage{hyperref}
\usepackage{xcolor}
\usepackage{color}
\usepackage{geometry}

\graphicspath{{./figures}}

\definecolor{backcolor}{HTML}{F2F2F2}

\lstdefinestyle{CPP}{
	language=C++,
	backgroundcolor=\color{backcolor},
	basicstyle=\footnotesize\ttfamily,
	breaklines=true,
	captionpos=b,
	commentstyle=\color{green},
	keywordstyle=\color{red},
	stringstyle=\color{red},
	showstringspaces=false,
	tabsize=2,
	frame=single,
	numbers=left,
	numbersep=5pt,
}

\begin{document}

\begin{titlepage}
	\begin{center}
		\includegraphics[scale=0.7]{logo.png}

		\vspace*{4cm}
		\textbf{Zaawansowane programowanie w języku C++}

		\vspace{1.5cm}
		\textit{Saper}

		\vspace{1.5cm}
		\textbf{Stanislau Antanovich\\Mykhailo Buzdyhan\\Vladyslav Gotovchykov}

		\vspace{4.5cm}
	\end{center}
\end{titlepage}

\tableofcontents

\newpage

\section{Realizacja}

\subsection{Klasa \textit{MyText}}
\lstinputlisting[style=CPP, firstline=4, lastline=37]{./code/MyText.h}

\subsubsection{Prywatne pola}
\begin{itemize}
	\item \textbf{string shribe}: Łańcuch znaków przechowujący początkowy tekts
	\item \textbf{Font font}: Obiekt klasy SFML \emph{Font}, który służy do ładowania i używania określonej czcionki
\end{itemize}
\subsubsection{Publiczne pola}
\begin{itemize}
	\item \textbf{sf::Text txt}: Obiekt klasy SFML \emph{Text}, który repzenetuje tekst do wyświetlenia
\end{itemize}
\subsubsection{Konstruktory}
\begin{itemize}
	\item \textbf{MyText(string shribeName)}
		\begin{itemize}
			\item \textbf{Cel}: Inicjalizuje obiekt \emph{MyText} z podanym łańcuchem znaków \textbf{shribeName}
			\item \textbf{Parametry}:
				\begin{itemize}
					\item \textbf{sting shribeName}; Początkowy tekst do wyświetlenia
				\end{itemize}
			\item \textbf{Operacje}
				\begin{itemize}
					\item Ładuje czcionkę z pliku ``\emph{fonts/bloodcrow.ttf}''
					\item Ustawia czcionkę, rozmiar znaków(20) oraz kolor wypełnienia dla obiektu \emph{text}
					\item Ustawia początkowy tekst na ``\emph{shribeName}''
				\end{itemize}
		\end{itemize}
	\item \textbf{MyText()}
		\begin{itemize}
			\item \textbf{Cel}: Konstruktor domyślny, który inicjalizuje obiekt \textbf{MyText} pustym łańcuchem znaków
			\item \textbf{Operacje}:
				\begin{itemize}
					\item Ładuje czcionkę z pliku ``\emph{fonts/bloodcrow.ttf}''
					\item Ustawia czcionkę, rozmiar znaków(20) oraz kolor wypełnienia dla obiektu \emph{text}
					\item Ustawia początkowy tekst na pusty łańcuch znaków
				\end{itemize}
		\end{itemize}
\end{itemize}
\subsubsection{Funkcje członkowskie}
\begin{itemize}
	\item \textbf{void sleditForSprite(Sprite\& s, float x, float y)}
		\begin{itemize}
			\item \textbf{Cel}: Ustawia pozycję tekstu względem pozycji podanego sprite'a \emph{s}
			\item \textbf{Parametry}:
				\begin{itemize}
					\item \textbf{Sprite\& s}: Referencja do obiektu klasy SFML \emph{Sprite}
					\item \textbf{float x}: Przesunięcie w osi X względem pozycji sprite'a
					\item \textbf{float y}: Przesunięcie w osi Y względem pozycji sprite'a
				\end{itemize}
			\item \textbf{Operacje}: Ustawia pozycję obiektu \emph{txt} na podstawie pozycji sprite'a oraz podanych przesunięć
		\end{itemize}
	\item \textbf{void getChislo(float n)}
		\begin{itemize}
			\item \textbf{Cel}: Dodaje wartość liczbową \emph{n} do istniejącego tekstu
			\item \textbf{Parametry}: \textbf{float n}: Liczba, która zostanie przekonwertowana na łańcuch znaków i dodana
			\item \textbf{Operacje}:
				\begin{itemize}
					\item Konwertuje liczbę \emph{n} na łańcuch znaków przy użyciu \emph{ostringstream}
					\item Ustawia tekst obiektu \emph{txt} na oryginalny łańcuch \emph{shribe} połączony z łańcuchem reprezentującym liczbę
				\end{itemize}
		\end{itemize}
	\item \textbf{void setString(string name)}
		\begin{itemize}
			\item \textbf{Cel}: Ustawia tekst na określony łańcuch znaków \emph{name}
			\item \textbf{Parametry}: \textbf{string name}: Nowy tekst do wyświetlenia
			\item \textbf{Operacje}: Ustawia tekst obiektu \emph{txt} na \emph{name}
		\end{itemize}
	\item \textbf{void setPosition(float x, float y)}
		\begin{itemize}
			\item \textbf{Cel}: Ustawia pozycję tekstu
			\item \textbf{Parametry}:
				\begin{itemize}
					\item \textbf{float x}: Współrzędna X pozycji tekstu
					\item \textbf{float y}: Współrzędna Y pozycji tekstu
				\end{itemize}
			\item \textbf{Operacje}: Ustawia pozycję obiektu \emph{txt} na (x, y)
		\end{itemize}
	\item \textbf{void setFillTextColor(float R, float G, float B)}
		\begin{itemize}
			\item \textbf{Cel}: Ustawia kolor tekstu
			\item \textbf{Parametry}:
				\begin{itemize}
					\item \textbf{float R}: Składowa czerwona koloru (0-255)
					\item \textbf{float G}: Składowa zielona koloru (0-255)
					\item \textbf{float B}: Składowa niebieska koloru (0-255)
				\end{itemize}
			\item \textbf{Operacje}: Ustawia kolor wypełnienia obiektu \emph{txt} przy użyciu wartości RGB
		\end{itemize}
	\item \textbf{void setCharacterSize(float a)}
		\begin{itemize}
			\item \textbf{Cel}: Ustawia rozmiar znaków tekstu
			\item \textbf{Parametry}: \textbf{float a}: Nowy rozmiar znaków
			\item \textbf{Operacje}: Ustawia rozmiar znaków obiektu \emph{txt} na \emph{a}
		\end{itemize}
	\item \textbf{void draw(RenderWindow\& window)}
		\begin{itemize}
			\item \textbf{Cel}: Rysuje tekst w podanym oknie renderowania
			\item \textbf{Parametry}:
				\begin{itemize}
					\item \textbf{RenderWindow\& window}: Referencja do obiektu klasy SFML \textbf{RenderWindow}
				\end{itemize}
			\item \textbf{Operacje}: Rysuje obiekt \emph{txt} w podanym oknie renderowania
		\end{itemize}

\end{itemize}
\subsubsection{Wnioski}
Klasa MyText kapsułkuje funkcjonalności związane z wyświetlaniem tekstu w bibliotece SFML, zapewniając metody do inicjalizacji, ustawiania pozycji, koloru, rozmiaru znaków oraz renderowania tekstu w graficznej aplikacji. Konstruktory ustawiają czcionkę oraz początkowy tekst, a funkcje członkowskie umożliwiają dynamiczne aktualizacje i wyświetlanie tekstu w oknie aplikacji.

\subsection{Klasa \textit{Button}}
\lstinputlisting[style=CPP, firstline=4]{./code/Button.h}
\subsubsection{Prywatne pola}
\begin{itemize}
	\item \textbf{float w, h}: Szerokość i wysokość przycisku
	\item \textbf{SoundBuffer buffer}: Bufor dźwięku używany do przechowywania danych dźwiękowych
	\item \textbf{Sound sound}: Obiekt dźwięku, który odtwarza dźwięk
	\item \textbf{bool press}: Flaga oznaczająca, czy przycisk jest wciśnięty
\end{itemize}
\subsubsection{Publiczne pola}
\begin{itemize}
	\item \textbf{RectangleShape button}: Obiekt kształtu prostokąta, który reprezentuje wizualny wygląd przycisku
\end{itemize}
\subsubsection{Konstruktory}
\begin{itemize}
	\item \textbf{Button(float W, float H, string shribeName)}
		\begin{itemize}
			\item \textbf{Cel}: Inicjalizuje obiekt Button z podanymi szerokością W, wysokością H oraz tekstem shribeName
			\item \textbf{Parametry}
				\begin{itemize}
					\item float W: Szerokość przycisku
					\item float H: Wysokość przycisku
					\item string shribeName: Tekst wyświetlany na przycisku
				\end{itemize}
			\item \textbf{Operacje}
				\begin{itemize}
					\item Wywołuje konstruktor klasy bazowej MyText z shribeName
					\item Ustawia szerokość w i wysokość h
					\item Ustawia flagę press na false
					\item Ustawia rozmiar prostokąta button
					\item Ustawia pozycję tekstu txt na pozycję prostokąta button
				\end{itemize}
		\end{itemize}
\end{itemize}
\subsubsection{Funkcje członkowskie}
\begin{itemize}
	\item \textbf{void sleditForSprite(Sprite\& s, float x, float y)}
		\begin{itemize}
			\item \textbf{Cel}: Ustawia pozycję przycisku względem pozycji podanego sprite'a s
			\item \textbf{Parametry}:
				\begin{itemize}
					\item Sprite\& s: Referencja do obiektu klasy SFML Sprite
					\item float x: Przesunięcie w osi X względem pozycji sprite'a
					\item float y: Przesunięcie w osi Y względem pozycji sprite'a
				\end{itemize}
			\item \textbf{Operacje}: Ustawia pozycję prostokąta button oraz tekstu txt na podstawie pozycji sprite'a oraz podanych przesunięć
		\end{itemize}
	\item \textbf{void draw(RenderWindow\& window)}
		\begin{itemize}
			\item \textbf{Cel}: Rysuje przycisk i tekst w podanym oknie renderowania
			\item \textbf{Parametry}: RenderWindow\& window: Referencja do obiektu klasy SFML RenderWindow
			\item \textbf{Operacje}:
				\begin{itemize}
					\item Ustawia pozycję tekstu txt na pozycję prostokąta button
					\item Rysuje prostokąt button oraz tekst txt w podanym oknie renderowania
				\end{itemize}
		\end{itemize}
	\item \textbf{bool pressed(Event\& event, Vector2i mousePosition)}
		\begin{itemize}
			\item \textbf{Cel}: Sprawdza, czy przycisk został naciśnięty
			\item \textbf{Parametry}:
				\begin{itemize}
					\item Event\& event: Referencja do obiektu klasy SFML Event
					\item Vector2i mousePosition: Pozycja myszy
				\end{itemize}
			\item \textbf{Operacje}:
				\begin{itemize}
					\item Sprawdza, czy przycisk został naciśnięty lewym przyciskiem myszy i aktualizuje flagę press
					\item  Zwraca true, jeśli przycisk został naciśnięty, w przeciwnym razie false
				\end{itemize}
		\end{itemize}
	\item \textbf{bool navediaMouse(Event\& event, Vector2i mousePosition)}
		\begin{itemize}
			\item \textbf{Cel}: Sprawdza, czy kursor myszy znajduje się nad przyciskiem
			\item \textbf{Parametry}:
				\begin{itemize}
					\item Event\& event: Referencja do obiektu klasy SFML Event
					\item Vector2i mousePosition: Pozycja myszy
				\end{itemize}
			\item \textbf{Operacje}: Zwraca true, jeśli kursor myszy znajduje się nad przyciskiem, w przeciwnym razie false
		\end{itemize}
	\item \textbf{void getSound(string failAudio)}
		\begin{itemize}
			\item \textbf{Cel}: Ładuje dźwięk z pliku
			\item \textbf{Parametry}: string failAudio: Ścieżka do pliku dźwiękowego
			\item \textbf{Operacje}:
				\begin{itemize}
					\item Ładuje dźwięk z pliku failAudio do bufora buffer
					\item Ustawia bufor buffer dla obiektu sound
				\end{itemize}
		\end{itemize}
	\item \textbf{void soundPlay()}
		\begin{itemize}
			\item \textbf{Cel}: Odtwarza dźwięk
			\item \textbf{Operacje}: Odtwarza dźwięk za pomocą obiektu sound
		\end{itemize}
	\item \textbf{void soundSetVolume(int volume)}
		\begin{itemize}
			\item \textbf{Cel}: Ustawia głośność dźwięku
			\item \textbf{Parametry}: int volume: Poziom głośności
			\item \textbf{Operacje}: Ustawia głośność dźwięku sound na wartość volume
		\end{itemize}
	\item \textbf{void setButtonSize(float W, float H)}
		\begin{itemize}
			\item \textbf{Cel}: Ustawia rozmiar przycisku
			\item \textbf{Parametry}:
				\begin{itemize}
					\item float W: Nowa szerokość przycisku
					\item float H: Nowa wysokość przycisku
				\end{itemize}
			\item \textbf{Operacje}
				\begin{itemize}
					\item Ustawia szerokość w i wysokość h
					\item Ustawia rozmiar prostokąta button
				\end{itemize}
		\end{itemize}
	\item \textbf{void setOringCenter()}
		\begin{itemize}
			\item \textbf{Cel}: Ustawia środek prostokąta button jako jego punkt oryginalny
			\item \textbf{Operacje}: Ustawia punkt oryginalny prostokąta button na (w / 2, h / 2)
		\end{itemize}
	\item \textbf{void setPosition(float x, float y)}
		\begin{itemize}
			\item \textbf{Cel}: Ustawia pozycję przycisku
			\item \textbf{Parametry}:
				\begin{itemize}
					\item float x: Współrzędna X pozycji przycisku
					\item float y: Współrzędna Y pozycji przycisku
				\end{itemize}
			\item \textbf{Operacje}: Ustawia pozycję prostokąta button na (x, y)
		\end{itemize}
	\item \textbf{void setFillRacktengelColor(float R, float G, float B)}
		\begin{itemize}
			\item \textbf{Cel}: Ustawia kolor wypełnienia prostokąta button
			\item \textbf{Parametry}:
				\begin{itemize}
					\item float R: Składowa czerwona koloru (0-255)
					\item float G: Składowa zielona koloru (0-255)
					\item float B: Składowa niebieska koloru (0-255)
				\end{itemize}
			\item \textbf{Operacje}: Ustawia kolor wypełnienia prostokąta button przy użyciu wartości RGB.
		\end{itemize}
\end{itemize}
\subsubsection{Wnioski}
Klasa Button rozszerza funkcjonalność klasy MyText o obsługę graficznych przycisków. Umożliwia tworzenie przycisków o określonych rozmiarach, pozycjach i kolorach, które mogą reagować na kliknięcia myszy. Dodatkowo, klasa ta obsługuje odtwarzanie dźwięków, co pozwala na dodanie efektów dźwiękowych do interakcji z przyciskiem.


\subsection{Plik \textit{Main}}
\lstinputlisting[style=CPP, firstline=4]{./code/main.cpp}

\subsubsection{Główne zmienne globalne}
\begin{itemize}
	\item const int MAXN = 17: Stała określająca maksymalny rozmiar siatki gry.
	\item int grid[MAXN][MAXN]{}: Siatka przechowująca rzeczywisty stan gry (czy w komórce jest mina).
	\item int sgrid[MAXN][MAXN]{}: Siatka widziana przez gracza (to, co gracz widzi na planszy).
	\item int used[MAXN][MAXN]{}: Siatka przechowująca informacje o tym, które komórki zostały już odkryte.
	\item int probabilities[MAXN][MAXN]: Siatka przechowująca obliczone prawdopodobieństwa wystąpienia min.
	\item int width, height: Szerokość i wysokość planszy (10x10).
	\item int numberOfFlags: Liczba flag umieszczonych przez gracza.
	\item bool dead: Flaga oznaczająca, czy gracz zginął.
	\item bool won: Flaga oznaczająca, czy gracz wygrał.
	\item int showProbs: Flaga określająca, czy mają być wyświetlane prawdopodobieństwa min.
	\item bool isFirstMoveMade: Flaga oznaczająca, czy został wykonany pierwszy ruch.
	\item float gameTime: Czas gry.
	\item int numberOfEmptyCells: Liczba pustych komórek na planszy.
	\item int flag: Flaga używana do sterowania zegarem.
	\item int bombChance: Prawdopodobieństwo wystąpienia miny.
\end{itemize}

\subsubsection{Obiekty przycisków}
\begin{itemize}
	\item \textbf{Button replayButton(105, 40, "Replay")}: Przycisk do rozpoczęcia nowej gry
	\item \textbf{Button aiProcentButton(50, 40, " AI")}: Przycisk do włączenia/wyłączenia wyświetlania prawdopodobieństw min.
\end{itemize}
\subsubsection{Funkcje}
\begin{itemize}
	\item clearAreaAround: Odkrywa komórki wokół danej komórki.
	\item openCell: Odkrywa daną komórkę i, jeśli jest pusta, odkrywa również komórki wokół niej.
	\item showDisplay: Wyświetla aktualny stan planszy i inne elementy interfejsu (np. licznik bomb, czas gry, komunikaty o wygranej/przegranej).
	\item dealWithEvent: Obsługuje zdarzenia (kliknięcia myszką, zamknięcie okna itp.).
	\item numberOfFlagsAround: Liczy liczbę flag wokół danej komórki.
	\item initializeGameField: Inicjalizuje planszę gry, losując położenie min.
	\item calculateNumberOfBombsAroundTheField: Oblicza liczbę min wokół każdej komórki na planszy.
	\item win: Ustawia stan gry na wygrany i oznacza wszystkie pozostałe komórki jako oznaczone flagą.
	\item initializeButtons: Inicjalizuje przyciski gry (ustawia ich wygląd i pozycję).
	\item calculateBombProbabilities: Oblicza prawdopodobieństwo wystąpienia min w nieodkrytych komórkach.
	\item calculateNumberAroundCell: Liczy liczbę komórek o danej wartości wokół określonej komórki.
	\item showProbabilities: Wyświetla prawdopodobieństwa wystąpienia min na planszy.
\end{itemize}

\subsubsection{Główna funkcja \textbf{main}}
\begin{enumerate}
	\item Inicjalizuje okno gry.
	\item Wczytuje tekstury dla kafelków i tła.
	\item Inicjalizuje przyciski i planszę gry.
	\item Obsługuje główną pętlę gry:
		\begin{itemize}
			\item Pobiera aktualną pozycję myszy.
			\item Jeśli pierwszy ruch nie został wykonany, resetuje licznik czasu.
			\item Jeśli gra została zakończona (wygrana lub przegrana), zatrzymuje licznik czasu.
			\item Obsługuje zdarzenia (np. kliknięcia myszką).
			\item Wyświetla aktualny stan gry
		\end{itemize}
\end{enumerate}
\subsubsection{Wnioski}
Ten kod stanowi kompletną implementację klasycznej gry w "Saper" z dodatkową funkcjonalnością wyświetlania prawdopodobieństw wystąpienia min oraz przyciskami do restartu gry i włączenia/wyłączenia wyświetlania prawdopodobieństw.

\end{document}
