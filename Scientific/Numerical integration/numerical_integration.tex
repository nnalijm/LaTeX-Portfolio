\documentclass[11pt, a4paper]{article}
\usepackage{mathtools}
\usepackage[polish]{babel}
\usepackage[T1]{fontenc}
\usepackage[utf8]{inputenc}
\usepackage[left=15mm, right=15mm, top=10mm, bottom=20mm]{geometry}
\usepackage{xcolor}
\usepackage{graphicx}
\usepackage{enumitem} \title{\textbf{Całkowanie numeryczne}}
\author{}
\date{}

\begin{document}

\maketitle

Metody obliczania całek(całki występują w trakcie obliczania pracy energii, momentów zgienających)


\begin{enumerate}
\item Metodę Newtona-Cotesa
\item Metodę Gaussa
\item Iteracyjny algorytm Romberga
\item Metodę Monte Carlo
\end{enumerate}

Całkę doprowadzamy do postaci standartowej

Wzory:

%\begin{align*}
\begin{gather*}
%\begin{split}
%\[
\gamma = \int_a^bf(x)dx = \int_{-1}^1F(\xi)d\xi\\
%\]
%\[
F(\xi) = \frac{b-a}{2}f\left(\frac{a+b}{2}+\frac{b-a}{2}\xi \right)
%\]
%\end{split}
\end{gather*}
%\end{align*}

(Zaletą zastosowania standaryzacji jest uniezależnienie sposobu obliczania całki od długości przedziału zmienności zmiennej niezależnej)

Następnie przedział <-1,1> dzielimy na n równych podprzedziałów i wyznaczamy w punktach podziału wartość funkcji F($\xi$). Następnie funkcję podcałkową zastępujemy wielomianem stopnia n-1. Wielomian ten całkujemy algebraicznie lub stosujemy równanie
\[
\gamma = \int_{-1}^1F(\xi)d\xi=\sum_{i=1}^nH_iF(\xi_i)
\]
gdzie współczynniki $H_i$ należy dobrać w zależności od stopnia wielomiana.

Najprostszy(ale i najmniej dokładny) wariant całkowania otrzymuje się, przyjmując n=2(tzw. metoda trapezów). Wtedy
\[
\gamma = F(-1) + F(1)
\]
gdy przyjęć n=3, otrzymujemy metodę Simpsona z regułą jednej trzeciej
\[
\gamma = \frac{1}{3}\left[F(-1)+4F(0)+F(1)\right]
\]
dla n=4 mamy metodę Simpsona z regułą trzech ósmych
\[
\gamma = \frac{1}{8}\left[F(-1)+3F(-\frac{1}{3})+3F(\frac{1}{3})+F(1)\right]
\]

\subsection*{Metoda trapezów - ogólny wzór trapezów}
\begin{gather*}
E\leq\frac{(b-a)^3}{12n^2}\cdot M \text{gdzie M} = max_{x\in<a,b>}|f''(x)|\\
\int_a^bf(x)dx\approx\sum_{i=1}^n\frac{y_{i-1}+y_i}{2}h=\frac{1}{2}h(y_0+y_1)+\\
+\frac{1}{2}h(y_1+y_2)+\dots+\frac{1}{2}h(y_{n-1}+y_1)=\\
=\frac{1}{2}h(y_0+2y_1+2y_2+\dots+2y_{n-1}+y_1)=\\
=\frac{1}{2}h(y_0+y_n+2\sum_{i=1}^{n-1}y_i)
\end{gather*}

\textbf{Zadanie 1}

Obliczyć wartość całki $\int_0^1\sqrt{1+x}$ stosując wzór ogólny trapezów z krokiem n=$\frac{1}{3}$.


Wynik analityczny to 1,21895

$h=\frac{b-a}{n}\rightarrow n=\frac{b-a}{n}=\frac{1-0}{\frac{1}{3}}=3$ - liczba podprzedziałów,
a liczba węzłów jest równa (n+1) czyli 4

Węzły mają wartości:
\begin{align*}
x_0&=0 - \text{lewy brzeg}\\
x_1&=x_0+h=\frac{1}{3}\\
x_2&=x_1+h=\frac{2}{3}\\
x_3&=x_2+h=1 - \text{lewy brzeg}
\end{align*}

Wzór na kwadraturę to

\begin{gather*}
y_0=\sqrt{1+x_0}=\sqrt{1+0}=1\\
\end{gather*}
\begin{align*}
\gamma_3=\sum_{i=1}^3\frac{y_{i-1}+y_i}{2}h&=\frac{1}{2}h\left(y_0+y_n+2\sum_{i=1}^{n-1}y_i\right)=\\
&=\frac{1}{2}h\left(y_0+y_3+2\sum_{i=1}^2y_i\right)\\
&=\frac{1}{2}h\left(1+\sqrt{1+1}+2\left(\sqrt{1+\frac{1}{3}}+\sqrt{1+\frac{2}{3}}\right)\right)\\
&=\frac{1}{2}\cdot\frac{1}{3}\left[1+\sqrt{2}+2\left(\sqrt{\frac{4}{3}}+\sqrt{\frac{5}{3}}\right)\right]=\\
&=1,2176
\end{align*}

Wartość błędu 
\[
E=\left|\frac{\gamma_3-\gamma}{\gamma}\right|\cdot100\%=\left|\frac{1,2176-1,21895}{1,21895}\right|\cdot100\%=11,08\%
\]


\textbf{Zadanie 2}

\subsection*{Ogólny wzór Simpsona}

\[
\int_a^bf(x)dx\approx\frac{1}{3}h\sum_{i=1}^{n-2}\left(y_{2i-2}+4y_{2i-1}+y_{2i}\right)=\frac{h}{3}\left[y_0+4(y_1+y_3+\dots+y_{n-1})+2(y_2+y_4+\dots+y_{n-2})+y_n\right]\\
\]
\[
\int_0^1\sqrt{1+x}dx
\]

Wartość kroku musi uleć zmianie bo przy h = $\frac{1}{3}$ ,,n`` nie jest liczbą parzystą.

Weźmy h=$\frac{1}{4}$, wtedy n=4. Węzły wyroszą $0, \frac{1}{4}, \frac{2}{4},\frac{3}{4}, 1$.

Wzór Simpsona

\begin{align*}
S_4&=\frac{1}{3}h\left(y_0+4y_1+y_2+y_2+4y_3+y_4\right)=\\
&=\frac{1}{3}h\left[y_0+4(y_1+y_3)+2y_2+y_4\right]=\\
&=\frac{1}{3}\cdot\frac{1}{4}\left[f(0)+4\left[f(\frac{1}{4})+f(\frac{3}{4})\right]+2f(\frac{1}{2})+f(1)\right]=\\
&=\frac{1}{12}\left[\sqrt{1}+4\left(\sqrt{\frac{5}{4}}+\sqrt{\frac{7}{4}}\right)+2\sqrt{\frac{3}{2}}+\sqrt{2}\right]=1,21895
\end{align*}

\textbf{Zadanie 3}

Oblicz wartość całki $\int_1^7e^xdx$ stosując wzór trapezów.

Użyj 6-ciu przedziałów h=$\frac{b-a}{n}=\frac{7-1}{6}=1$

$e^7-e^1=1093,914877$


\end{document}
