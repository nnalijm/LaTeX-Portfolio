\documentclass[10pt, a4paper]{article}

\usepackage{cancel}
\usepackage{tikz}
\usepackage{mdframed}
\usepackage{mathtools}
\usepackage{amsmath,amssymb}
\usepackage[polish]{babel}
\usepackage[T1]{fontenc}
\usepackage[utf8]{inputenc}
\usepackage{graphicx}
\usepackage[left=15mm,right=15mm, top=10mm, bottom=20mm]{geometry}
\usepackage{enumitem}

\title{\textbf{Interpolacja, aproksymacja}}
\author{}
\date{}

\begin{document}
\maketitle

Załóżmy, że w pewnym procesie fizycznym dysponujemy zbiorem pomiarów wartości danego parametru. Np. mierzymy charakterystykę rezystora i dokonujemy pomiarów prądu dla kilku punktów napięcia zasilającego. Nie mamy charakterystyki danej w sposób ciągły, nie znamy funkcji, które opisuje dany rezystor.

Interpolacją nazywamy postępowanie prowadzące do znalezienia wartości pewnej funkcji f(x) w dowolnym punkcie przedziału na podstawie znanych wartości tej funkcji w punktach $x_0, x_1,\dots,x_n$ zwanych węzłami interpolacji.

Postępowanie prowadzące do znalezienia wartości funkcji f(x) w punkcie leżącym poza przedziałem($x_0,x_1$) nazywamy ekstrapolacją

Interpolację | Ekstrapolację stosujemy gdy:

\begin{enumerate}
\item nie znamy samej funkcji f(x), a tylko jej wartość w pewnych punktach(tak przeważnie bywa w naukach doświedczalnych)
\item obliczenie wartości pewnej funkcji F(x) bezpośrednio z określającego ją wzoru ***(nastręcza?) zbyt duże trudności rachunkowe. Wtedy zastępujemy ją prostszą funkcją f(x), w której zakładamy, że w punktach $x_0, x_1, \dots, x_n$ ma te same wartości co funkcja F(x); w tym przypadku F(x) to f. interpolawana, a f(x) to f. interpolująca.
\end{enumerate}

\section*{Wielomian interpolacyjny Lagrange'a}

W przypadku gdy wielomian interpolujący na postać

\begin{align*}
W_n(x)&=y_0\frac{(x-x_1)(x-x_2)\dots(x-x_n)}{(x_0-x_1)(x_0-x_2)\dots(x_0-x_n)}+y_1\frac{(x-x_0)(x-x_2)\dots(x-x_b)}{(x_1-x_0)(x_1-x_2)\dots(x_1-x_n)}+\dots+\\
&+y_n\frac{(x-x_0)(x-x_1)\dots(x-x_{n-1})}{(x_n-x_0)(x_n-x_1)\dots(x_n-x_{n-1})}=\sum_{j=0}^ny_j\displaystyle\Pi_{\substack{k=0\\k\neq j}}^n\frac{(x-x_k)}{(x_j-x_k)}
\end{align*}
mamy do czynienia z interpolacją Logrange'a. Wielomian ten zeruje się we wszystkich punktach a=$x_1, x_2, \dots, x_n$=b co oznacza, że spełnia warunki interpolacji.

\textbf{Przykład}

Znaleźć wielomian interpoplacyjny Lagrange'a, który w punktach x=-2, -1, 0, 2 przyjmuje odpowiednio wartości 0,1,1,2.

Oblicz także przybliżoną wartość funkcji danej powyższymi wartościami w punkcie x=1.

\begin{mdframed}
Wielomian Langrange'a jest wielomianem stopnia co najwyżej n i jest jednoznacznym rozwiązaniem zadania interpolacyjnego dla dowolnego wyboru n+1 różnych węzłów interpolacji.
\end{mdframed}

\begin{center}
\begin{tabular}{|c|c|c|c|c|}
\hline
$x_i$ & -2 & -1 & 0 & 2\\
\hline
$y_i$ & 0 & 1 & 1 & 2 \\
\hline
\end{tabular}
\end{center}

\begin{align*}
W_3(x)&=0\frac{(x+1)(x-0)(x-2)}{(-2+1)(-2-0)(-2-2)}+1\frac{(x+2)(x-0)(x-2)}{(-1+2)(-1-0)(-1-2)}+1\frac{(x+2)(x+1)(x-2)}{(0+2)(0+1)(0-2)}+2\frac{(x+2)(x+1)(x-0)}{(2+2)(2+1)(2-0)}=\\
&=\frac{1}{3}x(x^2-4)-\frac{1}{4}(x^2-4)(x+1)+\frac{1}{12}x(x^2+3x+2)=\\
&=\frac{1}{3}(x^3-4x)-\frac{1}{4}(x^3+x^2-4x-4)+\frac{1}{12}(x^3+3x^2+2x)=\\
&=\textcolor{blue}{\frac{1}{3}x^3}-\textcolor{red}{\frac{4}{3}x}-\textcolor{blue}{\frac{1}{4}x^3}-\textcolor{cyan}{\frac{1}{4}x^2}+\textcolor{red}x+1+\textcolor{blue}{\frac{1}{12}x^3}+\textcolor{cyan}{\frac{1}{4}x^2}+\textcolor{red}{\frac{1}{6}x}=\\
&=\frac{1}{6}x^3-\frac{1}{6}x+1
\end{align*}

Używając tego wielomianu można teraz interpolować wartości funkcji f(x) w punktach przedziału <-2,2>. Przybliżona wartość funkcji f w punkcie x=1 wynosi: \[f(1)\approx W_3(1)=\frac{1}{6}\cdot1^3-\frac{1}{6}\cdot1+1=1\]

\textbf{Zadanie 1}

\begin{center}
\begin{tabular}{|c|c|c|c|c|}
\hline
$x_i$ & -2 & 0 & 1 & 3\\
\hline
$y_i$ & 1 & -1 & 2 & 1\\
\hline
\end{tabular}

$W_3(x)$=?\\
$W_3(2)$=?
\end{center}

\textbf{Zadanie 2}

\begin{center}
\begin{tabular}{|c|c|c|c|}
\hline
$x_i$ & 0 & 1 & 2\\
\hline
$y_i$ & 0 & 1 & 4\\
\hline
\end{tabular}

$W_2(x)$=?\\
$W_2(0,5)$=?
\end{center}

\section*{Interpolacja Newtona dla wielomianu dowolnego stopnia}

Zakładamy, że znamy wartości funkcji f(x) w co najmniej n+1 punktach $x_0, x_1,\dots, x_n$. Poszukujemy funkcje interpolującej w postaci wielomianu stopnia n:\[F(x)=b_0+b_1(x-x_0)+\dots+b_n(x-x_0)(x-x_1)\dots(x-x_{n-1})\]

Potrzebne wartości współczynników $b_0, b_1, \dots, b_n$ można obliczyć ze wzorów rekurencyjnych.

\begin{center}
\begin{tikzpicture}[main/.style, node distance=12.5mm]
\node[main] (1) {$f_4(x_4)$};
\node[main] (2) [above left of=1]{$f_3(x_3)$};
\node[main] (3) [below left of=1]{$f_3(x_4)$};
\node[main] (4) [above left of=2]{$f_2(x_2)$};
\node[main] (5) [below left of=3]{$f_2(x_4)$};
\node[main] (6) [below left of=2]{$f_2(x_3)$};
\node[main] (7) [above left of=5]{$f_1(x_3)$};
\node[main] (8) [above left of=6]{$f_1(x_2)$};
\node[main] (9) [above left of=4]{$f_1(x_1)$};
\node[main] (10) [above left of=9]{$x_0\quad f(x_0)$};
\node[main] (11) [above left of=8]{$x_1\quad f(x_1)$};
\node[main] (12) [above left of=7]{$x_2\quad f(x_2)$};
\node[main] (13) [below left of=5]{$f_1(x_4)$};
\node[main] (14) [below left of=7]{$x_3\quad f(x_3)$};
\node[main] (15) [below left of=13]{$x_4\quad f(x_4)$};
\draw (1) -- (2);
\draw (1) -- (3);
\draw (2) -- (4);
\draw (2) -- (6);
\draw (3) -- (5);
\draw (3) -- (6);
\draw (4) -- (8);
\draw (6) -- (8);
\draw (6) -- (7);
\draw (5) -- (7);
\draw (5) -- (13);
\draw (9) -- (10);
\draw (4) -- (9);
\draw (9) -- (11);
\draw (8) -- (12);
\draw (8) -- (11);
\draw (7) -- (12);
\draw (7) -- (14);
\draw (13) -- (14);
\draw (13) -- (15);
\end{tikzpicture}

\begin{tikzpicture}[main/.style, node distance=10mm]
\node[main] (1) {$-\frac{2}{3}$};
\node[main] (2) [above left of=1]{$\frac{11}{6}$};
\node[main] (3) [below left of=1]{$\frac{1}{2}$};
\node[main] (4) [above left of=2]{$-\frac{2}{3}$};
\node[main] (5) [above left of=3]{$\frac{1}{3}$};
\node[main] (6) [below left of=3]{0};
\node[main] (7) [above left of=4]{-2\quad 1};
\node[main] (8) [above left of=5]{0\quad -2};
\node[main] (9) [above left of=6]{1\quad 2};
\node[main] (10) [below left of=6]{3\quad 1};
\end{tikzpicture}
\end{center}

\begin{align*}
f_1(x_1)&=\frac{y_1-y_0}{x_1-x_0}=\frac{-2-1}{0+2}=-\frac{3}{2}\\
f_1(x_2)&=\frac{y_2-y_0}{x_2-x_0}=\frac{2-1}{1+2}=\frac{1}{3}\\
f_1(x_3)&=\frac{y_3-y_0}{x_3-x_0}=\frac{1-1}{3+2}=0\\
f_2(x_2)&=\frac{f_1(x_2)-f_1(x_1)}{x_2-x_1}=\frac{\frac{1}{3}-(-\frac{3}{2})}{1-0}=\frac{2}{6}+\frac{9}{6}=\frac{11}{6}\\
f_2(x_3)&=\frac{f_1(x_3)-f_1(x_1)}{x_3-x_1}=\frac{0-(-\frac{3}{2})}{3-0}=\frac{\frac{3}{2}}{3}=\frac{1}{2}\\
f_3(x_3)&=\frac{f_2(x_3)-f_2(x_3)}{x_3-x_2}=\frac{\frac{1}{2}-\frac{11}{6}}{3-1}=\frac{1}{2}\left(\frac{3}{6}-\frac{11}{6}\right)=\frac{1}{2}\cdot\left(-\frac{8}{6}\right)=-\frac{8}{12}=-\frac{2}{3}
\end{align*}

\begin{align*}
W_3(x)&=1-\frac{3}{2}(x+2)+\frac{11}{6}(x+2)(x-0)-\frac{2}{3}(x+2)(x-0)(x-1)=\\
&=1-\textcolor{red}{\frac{3}{2}x}-3+\textcolor{cyan}{\frac{11}{6}x^2}+\textcolor{red}{\frac{11}{3}x}-\frac{2}{3}(x^3+x^2-2x)=\\
&=1-\textcolor{red}{\frac{3}{2}x}-3+\textcolor{cyan}{\frac{11}{6}x^2}+\textcolor{red}{\frac{11}{3}x}-\textcolor{blue}{\frac{2}{3}x^3}-\textcolor{cyan}{\frac{2}{3}x^2}+\textcolor{red}{\frac{4}{3}x}=\\
&=\textcolor{blue}{-\frac{2}{3}x^3}+\textcolor{cyan}{\frac{7}{6}x^2}+\textcolor{red}{\frac{21}{6}x}-2-\frac{3}{2}+\frac{4}{3}=\frac{-9+8}{6}=-\frac{1}{6}+\frac{22}{6}
\end{align*}

\textbf{Zadanie}

\begin{center}
\begin{tabular}{c c}
-2& 0\\
-1& 1\\
0& 1\\
2& 2
\end{tabular}
\end{center}
\[W_3(x)=\dots\]

\section*{Interpolacja odwrotna}
Traktując $y_i$ jako węzły interpolacji, a $x_i$ jako wartości funkcji w tych węzłach ($x_i=g(y_i)$) można utworzyć wielomian interpolujący $f$ odwrotną do $f(x)$.

Korzystając z obliczonego w ten sposób wielomianu można znaleźć $x_p$, dla którego funkcje(x) przyjmuje z góry zadaną wartość $y_p$.(Korzystne przy szukaniu miejsc zerowych)

\textbf{Przykład}

Znaleźćp przybliżenie miejsca zerowego równania $x-sin(x)=0$ w przedziale $x\in<\frac{\Pi}{2},\frac{3}{2}\Pi>$

W podobnym przedziale wprowadzimy 3 węzły obliczając wartości węzłowe na podstawie równania $f(x)=x-sin(x)$

\begin{center}
\begin{tabular}{|c|c|c|c|}
\hline
&&&\\
$i$ & 1 & 2 & 3\\
&&&\\
\hline
&&&\\
$x_i$ & $\frac{\Pi}{2}$&$\Pi$&$\frac{3}{2}\Pi$\\
&&&\\
\hline
&&&\\
$f_i=f(x_i)$&$\frac{\Pi}{2}-1$&$\Pi$&$\frac{3}{2}\Pi+1$\\
&&&\\
\hline
\end{tabular}
\end{center}

\begin{gather*}
W_n(y)=x_0\frac{(y-y_1)(y-y_2)\dots(y-y_n)}{(y_0-y_1)(y_0-y_2)\dots(y_0-y_n)}+\dots+x_n\frac{(y-y_0)(y-_1)\dots(y-y_{n-1})}{(y_n-y_0)(y_n-y_1)\dots(y_n-y_{n-1})}\\
W_3(y)=\frac{\Pi}{2}\frac{(y-\Pi)(y-\frac{3}{2}-1)}{(\frac{\Pi}{2}-1-\Pi)(\frac{\Pi}{2}-1-\frac{3}{2}\Pi-1)}+\Pi\frac{(y-\frac{\Pi}{2}+1)(y-\frac{3}{2}\Pi-1)}{(\Pi-\frac{\Pi}{2}+1)}+\frac{3}{2}\Pi\frac{(y-\Pi)(y-\frac{\Pi}{2}+1)}{(\frac{3}{2}\Pi+1-\Pi)(\frac{3}{2}\Pi+1-\frac{\Pi}{2}+1)}\\
W_3(y)=\frac{2\Pi}{2(\Pi+2)^2}(y-\Pi)(y-\frac{3}{2}\Pi-1)-\frac{4\Pi}{(2+\Pi)^2}(y-\frac{\Pi}{2}+1)(y-\frac{3}{2}\Pi-1)+\\
+\frac{3}{2}\Pi\frac{2}{(\Pi+2)^2}(y-\Pi)(y-\frac{\Pi}{2}+1)=\Pi\frac{y+2}{\Pi+2}
\end{gather*}

Przybliżenie miejsca zerowego równania \[W_3(0)=\Pi\frac{0+2}{2+\Pi}=\frac{2\Pi}{2+\Pi}\]

\section*{Aproksymacja}

Dla poniższych danych znaleźć aproksymację liniową. Rozpatrzymy 2 przypadki: metodę zwykłą i ważoną przyjmując każdemu z węzłów jego numer jako wagę.

\begin{center}
\begin{tabular}{|c|c|c|c|}
\hline
&&&\\
$i$&1&2&3\\
&&&\\
\hline
&&&\\
$x_i$&0&1&2\\
&&&\\
\hline
&&&\\
$f_i$&0&1&4\\
&&&\\
\hline
\end{tabular}
\end{center}

Przyjmujemy f liniową $p(x)=ax+b$

\subsection*{Metoda zwykła}

Układamy funkcjonał \[B(a,b)=\sum_{i=1}^3(a-x_i+b-f_i)^2=(a\cdot0+b-0)^2+(a\cdot1+b-1)^2+(a\cdot2+b-4)^2\]

Różniczkujemy po zmiennych a i b.

\begin{gather*}
\begin{cases}
\frac{\partial}{\partial a}B(a,b)=2(a+b-1)+2\cdot 2(2a+b-4)=0\\
\frac{\partial}{\partial b}B(a,b)=2b+2(a+b-1)+2(2a+b-4)=0
\end{cases}
\Longleftrightarrow
\begin{cases}
5a+3b=9\\
3a+3b=5
\end{cases}
\end{gather*}

\begin{gather*}
a=2\\
b=-\frac{1}{3}\Rightarrow p(x)=2x-\frac{1}{3}
\end{gather*}

\subsection*{Metoda ważona}

Wagi: $w+1=1; w_2=2; w_3=3$

\[B(a,b)=\sum_{i=1}^3w_i(a-x_i+b-f_i)^2=(a\cdot0+b-0)^2+2(a\cdot1+b-1)^2+3(a\cdot2+b-4)^2\]

Różniczkujemy po zmiennych a i b:

\begin{gather*}
\begin{cases}
\frac{\partial}{\partial a}B(a,b)=2\cdot2(a+b-1)+2\cdot2\cdot3(2a+b-4)=0\\
\frac{\partial}{\partial b}B(a,b)=1\cdot2\cdot b+2\cdot2(a+b-1)+2\cdot 3(2a+b-4)=0
\end{cases}\\
\begin{cases}
14a+8b=26\\
8a+6b=14
\end{cases}
\Longleftrightarrow
\begin{cases}
a=2,2\\
b=-0,6
\end{cases}
\Rightarrow
\begin{cases}
p(x)=2.2x-0,6
\end{cases}
\end{gather*}

\textbf{Zadanie}

\begin{center}
\begin{tabular}{|c|c|c|c|c|}
\hline
&&&&\\
$i$&1&2&3&4\\
&&&&\\
\hline
&&&&\\
$x_i$&1&3&4&6\\
&&&&\\
\hline
&&&&\\
$f_i$&1&2&4&4\\
&&&&\\
\hline
\end{tabular}

Metoda zwykła
\end{center}

\begin{center}
\begin{tabular}{c c c c c}
$i$&1&2&3&4\\
$x_i$&1&3&4&6\\
$y_i$&1&2&4&4
\end{tabular}
\end{center}

\begin{gather*}
B(a,b)=\sum_{i=1}^4(ax_i+b-y_i)^2=(a+b-1)^2+(3a+b-2)^2+(4a+b-4)^2+(6a+b-4)^2\\
\begin{cases}
\frac{\partial B}{\partial a}=2(a+b-1)+6(3a+b-2)+8(4a+b-4)+12(6a+b-4)=0\\
\frac{\partial B}{\partial b}=2(a+b-1)+2(3a+b-2)+2(4a+b-4)+2(6a+b-4)=0
\end{cases}\\
\begin{cases}
\textcolor{red}{2a}+\textcolor{cyan}{2b}-2+\textcolor{red}{18a}+\textcolor{cyan}{6b}-12+\textcolor{red}{32a}+\textcolor{cyan}{8b}-32+\textcolor{red}{72a}+\textcolor{cyan}{12b}-48=0\\
\textcolor{red}{2a}+\textcolor{cyan}{2b}-2+\textcolor{red}{6a}+\textcolor{cyan}{2b}-4+\textcolor{red}{8a}-\textcolor{cyan}{2b}-8+\textcolor{red}{12a}-\textcolor{cyan}{2b}-8=0
\end{cases}\\
\begin{cases}
124a+28b=94 \quad|:2\\
28a+8b=22 \quad|:2
\end{cases}
\begin{cases}
62a+14b=47\\
14a+4b=11
\end{cases}\\
\end{gather*}

\begin{gather*}
4b=11-14a\\
b=\frac{11-14a}{4}\\
\end{gather*}

\begin{align*}
62a&+\cancel{14}\left(\frac{11-14a}{\cancel{4}}\right)=47 & b&=\frac{11}{4}-\frac{\cancel{14}^7}{2}\cdot\frac{17}{\cancel{26}13}\\
62a&+\frac{77}{2}-\frac{98}{2}a=47 &b&=\frac{11}{4}-\frac{119}{52}\\
62a&+\frac{77}{2}-49a=47 &b&=\frac{13\cdot11-119}{52}=\frac{24}{54}\\
13a&=\frac{94-77}{2} & b&=\frac{6}{13}\\
13a&=\frac{17}{2}\quad|\cdot\frac{1}{13}\\
\end{align*}


\begin{center}
\begin{tabular}{c c c c}
$i$&1&2&3\\
$x_i$&0&1&2\\
$f_i$&0&1&4
\end{tabular}
\end{center}

Funkcja aproksymująca $p(x)=ax^2+bx+c$

\[B(a,b,c)=\sum_{i=1}^3(ax_i^2+bx_i+c-f_i)^2=(c-0)^2+(a+b+c-1)^2+(4a+2b+c-4)^2\]

\begin{gather*}
\begin{cases}
\frac{\partial B}{\partial a}=(a+b+c-1)+4(4a+2b+c-4)=0\\
\frac{\partial B}{\partial b}=(a+b+c-1)+2(4a+2b+c-4)=0\\
\frac{\partial B}{\partial c}=c+(a+b+c-1)+(4a+2b+c-4)=0
\end{cases}\\
\begin{cases}
17a+9b+5c=17\\
9a+5b+3c=9\\
5a+3b+3c=5
\end{cases}
\Rightarrow
\begin{cases}
a=1\\
b=0\\
c-0
\end{cases}
\Rightarrow
\begin{cases}
p(x)=x^2
\end{cases}
\end{gather*}

Jest to przypadek szczególny: budowanie aproksymacji kwadratowej na trzech węzłach daje w rezultacie interpolację: otrzymaliśmy wyjściową parabolę. Meotody ważonej nie ma sensu stosować.
\end{document}
