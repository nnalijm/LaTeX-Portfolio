\documentclass[9pt, a4paper]{article}

\usepackage{mathtools}
\usepackage{amsmath,amssymb}
\usepackage[left=10mm, right=20mm, top=20mm, bottom=20mm]{geometry}
\usepackage{cancel}
%\usepackage{mdframed}
\usepackage{xcolor}
\usepackage{graphicx}
\usepackage[T1]{fontenc}
\usepackage[polish]{babel}
\usepackage[utf8]{inputenc}



\title{\textbf{\Large FEM}\\Classical formulation - sformułowanie klasyczne}
\author{}
\date{}

\begin{document}

\maketitle

Równanie: 

\begin{align*}
EA\frac{\partial^2u(x,t)}{\partial x^2}&-\rho A\frac{\partial^2(x,t)}{\partial t^2}=0\quad|\quad:\rho A\\
\frac{E\cancel{A}}{\rho\cancel{A}}\frac{\partial^2u(x,t)}{\partial x^2}&-\frac{\partial^2u(x,t)}{\partial t^2}=0\\
c^2\frac{\partial^2u(x,t)}{\partial x^2}&=\frac{\partial^2u(x,t)}{\partial t^2}\qquad c=\sqrt{\frac{E}{\rho}}
\end{align*}

Metoda rozdzielania zmiennych: zakładamy rozwiązanie postaci

\begin{align*}
u(x_1t)&=U(x)T(t)\\
\frac{\partial^2u(x,t)}{\partial x^2}=\frac{d^2U(x)}{dx^2}T(t) &\qquad\frac{\partial^2u(x,t)}{\partial t}=U(x)\frac{d^2T(t)}{dt^2}\\
c^2\frac{d^2U(x)}{dx^2}T(t)&=U(x)\frac{d^2T(t)}{dt^2}\quad|\quad :U\quad|:T\\
\frac{c^2}{U(x)}\frac{d^2U(x)}{dx^2}&=\frac{1}{T(t)}\frac{d^2T(t)}{dt^2}=-\omega^2\\
\frac{c^2}{U(x)}\frac{d^2U(x)}{dx^2}&=-\omega^2\\
\frac{c^2}{U(x)}\frac{d^2U(x)}{dx^2}&+\omega^2=0\quad|\quad\cdot\frac{U(x)}{c^2}\\
\end{align*}

\begin{center}
\begin{tabular}{||c||}
\hline
\\
\(\frac{d^2U(x)}{dx^2}+\frac{\omega^2}{c^2}U(x)=0\)\\
\\
\hline
\end{tabular}

\[\color{cyan}\frac{\omega^2}{c^2}=k^2\]
\end{center}


\[\Delta u+k^2u=0\]

Zakładamy rozwiązanie przybliżone:

\begin{align*}
\tilde{u}(x)&=\sum_{\nu}a_\nu\varphi_\nu(x)\\
D\tilde{u}(x)&=\sum_\nu a_\nu D\varphi_\nu(x)\\
D^2\tilde{u}(x)&=\sum_\nu a_\nu D^2\varphi_\nu(x) 
\end{align*}

Podstawiamy do równania: 

\begin{align*}
\sum_\nu a_\nu D^2\varphi_\nu(x)&+k^2\sum_\nu a_\nu\varphi_\nu(x)=0
\end{align*}
\clearpage
Dyskretyzacja obszaru: 2 elementy, 5 węzłów

Funkcje kształtu : stopień 2

\begin{center}
\includegraphics[scale=0.2]{shape_fun.png}
\end{center}

\begin{align*}
\varphi_1(x)&=\frac{(x-x_2)(x-x_3)}{(x_1-x_2)(x_1-x_3)}\\
\varphi_2(x)&=\frac{(x-x_1)(x-x_3)}{(x_2-x_1)(x_2-x_3)}\\
\varphi_3(x)&=\frac{(x-x_1)(x-x_2)}{(x_3-x_1)(x_3-x_2)}
\end{align*}

Macierze lokalne

Element 1

\begin{align*}
a_1^{\textcolor{red}{(1)}}[D^2\varphi_1^{\textcolor{red}{(1)}}(x_1)+k^2\varphi_1^{\textcolor{red}{(1)}}(x_1)]+a_2^{\textcolor{red}{(1)}}[D^2\varphi_2^{\textcolor{red}{(1)}}(x_1)+k^2\varphi_2^{\textcolor{red}{(1)}}(x_1)]+a_3^{\textcolor{red}{(1)}}[D^2\varphi_3^{\textcolor{red}{(1)}}(x_1)+k^2\varphi_3^{\textcolor{red}{(1)}}(x_1)]=0\\
a_1^{\textcolor{red}{(1)}}[D^2\varphi_1^{\textcolor{red}{(1)}}(x_2)+k^2\varphi_1^{\textcolor{red}{(1)}}(x_2)]+a_2^{\textcolor{red}{(1)}}[D^2\varphi_2^{\textcolor{red}{(1)}}(x_2)+k^2\varphi_2^{\textcolor{red}{(1)}}(x_2)]+a_3^{\textcolor{red}{(1)}}[D^2\varphi_3^{\textcolor{red}{(1)}}(x_2)+k^2\varphi_3^{\textcolor{red}{(1)}}(x_2)]=0\\
a_1^{\textcolor{red}{(1)}}[D^2\varphi_1^{\textcolor{red}{(1)}}(x_3)+k^2\varphi_1^{\textcolor{red}{(1)}}(x_3)]+a_2^{\textcolor{red}{(1)}}[D^2\varphi_2^{\textcolor{red}{(1)}}(x_3)+k^2\varphi_2^{\textcolor{red}{(1)}}(x_3)]+a_3^{\textcolor{red}{(1)}}[D^2\varphi_3^{\textcolor{red}{(1)}}(x_3)+k^2\varphi_3^{\textcolor{red}{(1)}}(x_3)]=0
\end{align*}

Element 2

\begin{align*}
a_1^{\textcolor{red}{(2)}}[D^2\varphi_1^{\textcolor{red}{(2)}}(x_1)+k^2\varphi_1^{\textcolor{red}{(2)}}(x_1)]+a_2^{\textcolor{red}{(2)}}[D^2\varphi_2^{\textcolor{red}{(2)}}(x_1)+k^2\varphi_2^{\textcolor{red}{(2)}}(x_1)]+a_3^{\textcolor{red}{(2)}}[D^2\varphi_3^{\textcolor{red}{(2)}}(x_1)+k^2\varphi_3^{\textcolor{red}{(2)}}(x_1)]=0\\
a_1^{\textcolor{red}{(2)}}[D^2\varphi_1^{\textcolor{red}{(2)}}(x_2)+k^2\varphi_1^{\textcolor{red}{(2)}}(x_2)]+a_2^{\textcolor{red}{(2)}}[D^2\varphi_2^{\textcolor{red}{(2)}}(x_2)+k^2\varphi_2^{\textcolor{red}{(2)}}(x_2)]+a_3^{\textcolor{red}{(2)}}[D^2\varphi_3^{\textcolor{red}{(2)}}(x_2)+k^2\varphi_3^{\textcolor{red}{(2)}}(x_2)]=0\\
a_1^{\textcolor{red}{(2)}}[D^2\varphi_1^{\textcolor{red}{(2)}}(x_3)+k^2\varphi_1^{\textcolor{red}{(2)}}(x_3)]+a_2^{\textcolor{red}{(2)}}[D^2\varphi_2^{\textcolor{red}{(2)}}(x_3)+k^2\varphi_2^{\textcolor{red}{(2)}}(x_3)]+a_3^{\textcolor{red}{(2)}}[D^2\varphi_3^{\textcolor{red}{(2)}}(x_3)+k^2\varphi_3^{\textcolor{red}{(2)}}(x_3)]=0
\end{align*}

\[\footnotesize
A=
\begin{bmatrix}
\color{red}D^2\varphi_1^{(1)}(x_1)+k^2\varphi_1^{(1)}(x_1)&\color{red}D^2\varphi_2^{(1)}(x_1)+\cancel{k^2\varphi_2^{(1)}(x_1)}&\color{red}D^2\varphi_3^{(1)}(x_1)+\cancel{k^2\varphi_3^{(1)}(x_1)}&0&0\\
\color{red}D^2\varphi_1^{(1)}(x_2)+\cancel{k^2\varphi_1^{(1)}(x_2)}&\color{red}D^2\varphi_2^{(1)}(x_2)+k^2\varphi_2^{(1)}(x_2)&\color{red}D^2\varphi_3^{(1)}(x_2)+\cancel{k^2\varphi_3^{(1)}(x_2)}&0&0\\
\color{red}D^2\varphi_1^{(1)}(x_3)+\cancel{k^2\varphi_1^{(1)}(x_3)}&\color{red}D^2\varphi_2^{(1)}(x_3)+\cancel{k^2\varphi_2^{(1)}(x_3)}&\color{red}D^2\varphi_3^{(1)}(x_3)+k^2\varphi_3^{(1)}(x_3)+&\color{cyan}D^2\varphi_4^{(2)}(x_3)+\cancel{k^2\varphi_4^{(2)}(x_3)}&\color{cyan}D^2\varphi_5^{(2)}(x_3)+\cancel{k^2\varphi_5^{(2)}(x_3)}\\
& & \color{cyan}+D^2\varphi_3^{(2)}(x_3)+k^2\varphi_3^{(2)}(x_3)& &\\
0&0&\color{cyan}D^2\varphi_3^{(2)}(x_4)+\cancel{k^2\varphi_3^{(2)}(x_4)}&\color{cyan}D^2\varphi_4^{(2)}(x_4)+k^2\varphi_4^{(2)}(x_4)&\color{cyan}D^2\varphi_5^{(2)}(x_4)+\cancel{k^2\varphi_5^{(2)}(x_4)}\\
0&0&\color{cyan}D^2\varphi_3^{(2)}(x_5)+\cancel{k^2\varphi_3^{(2)}(x_5)}&\color{cyan}D^2\varphi_4^{(2)}(x_5)+\cancel{k^2\varphi_4^{(2)}(x_5)}&\color{cyan}D^2\varphi_5^{(2)}(x_5)+k^2\varphi_5^{(2)}(x_5)
\end{bmatrix}
\]
\[
RHS=
\begin{bmatrix}
0\\
0\\
0\\
0\\
0
\end{bmatrix}
\]

\begin{center}
\begin{tabular}{ c | c }
lokalnie & globalne\\
\hline
1 1 & 1\\
1 2 & 2\\
\color{cyan}1 3 &\color{cyan}3\\
\hline
\color{cyan}2 1 &\color{cyan}3\\
2 2 & 4\\
2 3 & 5\\
\hline
\end{tabular}
\end{center}

\newpage
\section*{Uwzględnienie warunków brzegowych}
\begin{center}
\begin{tabular}{|| c | c ||}
\hline
\(U|_{x=0}=0\) & \(\frac{du}{dx}|_{x=l}=0\)\\
Element 1 & Element 2\\
Warunek Dirichleta & Warunek Neumanna\\
\hline
\end{tabular}
\end{center}


\textcolor{blue}{Warunek Dirichleta}
\[U|_{x=0}=0\to a_1\varphi_1^{\textcolor{red}{1}}(x_1)+\cancel{a_2\varphi_2^{\textcolor{red}{1}}(x_1)}+\cancel{a_3\varphi_3^{\textcolor{red}{1}}(x_1)}=0\]

\textcolor{blue}{\emph{Funkcje $\varphi_2^1(x_1)\quad i\quad\varphi_3^1(x_1)$ są równe 0 w pierwszym węźle}}
\hfill\break

\textcolor{red}{Warunek Neumanna}

\[\frac{du}{dx}|_{x=l}=0\to a_3D\varphi_3^{\textcolor{red}{2}}(x_5)+a_4D\varphi_4^{\textcolor{red}{2}}(x_5)+a_5d\varphi_5^{\textcolor{red}{2}}(x_5)=0\]

\textcolor{red}{\emph{Wszystkie pochodne mają wartość $\ne$ 0}}

* \emph{W pierwszym węźle równanie \underline{\underline{nie jest}} spełnione, obowiązuje \underline{\underline{warunek brzegowy}}.}

* \emph{W ostatnim węźle równanie \underline{\underline{nie jest}} spełnione, obowiązuje \underline{\underline{warunek brzegowy}}.}


Dlatego równanie dla pierwszego i ostatniego węzła(węzły brzegowe) zastępujemy równaniami warunków brzegowych:

Pierwszy wiersz macierzy A:
\[
\begin{bmatrix}
\varphi_1^{\textcolor{red}{1}}(x_1)&0&0&0&0
\end{bmatrix}
\]

Ostatni wiersz macierzy A:
\[
\begin{bmatrix}
0&0&D\varphi_3^{\textcolor{red}{2}}(x_5)&D\varphi_4^{\textcolor{red}{2}}(x_5)&D\varphi_5^{\textcolor{red}{2}}(x_5)
\end{bmatrix}
\]

\begin{align*}
\begin{rcases}
%1
\begin{rcases}
\varphi_1^{(1)}(x)=\frac{(x-x_2)(x-x_3)}{(x_1-x_2)(x_1-x_3)}\\
D^2\varphi_1^{(1)}(x)=\frac{2}{(x_1-x_2)(x_1-x_3)}=\frac{2}{(0-0.25)(0-0.5)}=16
\end{rcases}
\text{\textcolor{red}{węzeł 1}}\\
%2
\begin{rcases}
\varphi_2^{(1)}(x)=\frac{(x-x_1)(x-x_3)}{(x_2-x_1)(x_2-x_3)}\\
D^2\varphi_2^{(1)}(x)=\frac{2}{(x_2-x_1)(x_2-x_3)}=\frac{2}{(0.25-0)(0.25-0.5)}=-32
\end{rcases}
\text{\textcolor{red}{węzeł 2}}\\
%3
\begin{rcases}
\varphi_3^{(1)}(x)=\frac{(x-x_1)(x-x_2)}{(x_3-x_1)(x_3-x_2)}\\
D^2\varphi_3^{(1)}(x)=\frac{2}{(x_3-x_1)(x_3-x_2)}=\frac{2}{(0.5-0)(0.5-0.25)}=16
\end{rcases}
\text{\textcolor{red}{węzeł 3}}\\
\end{rcases}
\text{\textcolor{cyan}{Element 1}}
\end{align*}

\begin{align*}
\begin{rcases}
\begin{rcases}
\varphi_3^{(2)}(x)=\frac{(x-x_4)(x-x_5)}{(x_3-x_4)(x_3-x_5)}\\
D\varphi_3^{(2)}(x)=\frac{2x-x_4-x_5}{(x_3-x_4)(x_3-x_5)}\\
D^2\varphi_3^{(2)}(x)=\frac{2}{(x_3-x_4)(x_3-x_5)}=\frac{2}{(0.5-0.75)(0.5-1)}=16
\end{rcases}
\text{\textcolor{red}{węzeł 3}}\\
\begin{rcases}
\varphi_4^{(2)}(x)=\frac{(x-x_3)(x-x_5)}{(x_4-x_3)(x_4-x_5)}\\
D\varphi_4^{(2)}(x)=\frac{2x-x_3-x_5}{(x_4-x_3)(x_4-x_5)}\\
D^2\varphi_4^{(2)}(x)=\frac{2}{(x_4-x_3)(x_4-x_5)}=\frac{2}{(0.75-0.5)(0.75-1)}=-32
\end{rcases}
\text{\textcolor{red}{węzeł 4}}\\
\begin{rcases}
\varphi_5^{(2)}(x)=\frac{(x-x_3)(x-x_4)}{(x_5-x_3)(x_5-x_4)}\\
D\varphi_5^{(2)}(x)=\frac{2x-x_3-x_4}{(x_5-x_3)(x_5-x_4)}\\
D^2\varphi_5^{(2)}(x)=\frac{2}{(x_5-x_3)(x_5-x_4)}=\frac{2}{(1-0.5)(1-0.75)}=16
\end{rcases}
\text{\textcolor{red}{węzeł 5}}
\end{rcases}
\text{\textcolor{cyan}{Element 2}}
\end{align*}

\newpage

\[
A=
\begin{bmatrix}
&&\textcolor{blue}{Warunek\quad Dirichleta}&&\\
\color{blue}\varphi_1^1(x_1)&\color{blue}0&\color{blue}0&\color{blue}0&\color{blue}0\\
D^2\varphi_1^1(x_2)&D^2\varphi_2^1(x_2)+k^2\varphi_2^1(x_2)&D^2\varphi_3^1(x_2)&0&0\\
D^2\varphi_1^1(x_3)&D^2\varphi_2^1(x_3)&D^2\varphi_3^1(x_3)+k^2\varphi_3^1(x_3)+D^2\varphi_3^2(x_3)+k^2\varphi_3^2(x_3)&D^2\varphi_4^2(x_3)&D^2\varphi_5^2(x_3)\\
0&0&D^2\varphi_3^2(x_4)&D^2\varphi_4^2(x_4)+k^2\varphi_4^2(x_4)&D^2\varphi_5^2(x_4)\\
\color{red}0&\color{red}0&\color{red}D\varphi_3^2(x_5)&\color{red}D\varphi_4^2(x_5)&\color{red}D\varphi_5^2(x_5)\\
&&\textcolor{red}{Warunek\quad Neumanna}&&
\end{bmatrix}
\]

\[A=
\begin{bmatrix}
1&0&0&0&0\\
16&-32+k^2\cdot1=-29.4957&16&0&0\\
16&-32&16+k^2\cdot1+16+k^2=37.0087&-32&16\\
0&0&16&-32+k^2\cdot1=-29.4957&16\\
0&0&2&-8&6
\end{bmatrix}
\]

Dane: materiał: aluminium, \(E=69\cdot10^9PQ, \rho=2700\frac{kg}{m^3}, c=\sqrt{\frac{E}{\rho}}, \omega=8000, k^2=\frac{\omega^2}{c^2}=2.5043\)

\hfill \break

Wymuszenie

Siła punktowa w $x_3$

Przykładowe równanie dla węzła $x_2$:

\begin{align*}
a_1D^2\varphi_1^{(1)}(x_2)+a_2(D^2\varphi_2^{(1)}(x_2)+k^2\varphi_2^{(1)}(x_2))+a_3D^2\varphi_3^{(1)}(x_2)=0
\end{align*}

Dla $x_3$:

\begin{align*}
a_1D^2\varphi_1^{(1)}(x_3)+a_2D^2\varphi_2^{(1)}(x_3)&+a_3[D^2\varphi_3^{(1)}(x_3)+k^2\varphi_3^{(1)}(x_3)+D^2\varphi_3^{(2)}(x_3)+k^2\varphi_3^{(2)}(x_3)]\\
&+a_4D^2\varphi_4^{(2)}(x_3)+a_5D^2\varphi_5^{(2)}(x_3)=f
\end{align*}

Założenie: f=1N

Wektor wymuszeń (RHS):
\[b=
\begin{bmatrix}
0\\0\\1\\0\\0
\end{bmatrix}
\]

Rozwiązanie URL(za pomocą MATLABA): A\textbackslash b
\[Q=
\begin{bmatrix}
0\\1.4370\\2.6492\\3.4620\\3.7329
\end{bmatrix}
\]

Rozwiązanie:
\[\tilde{u}(x)=a_1\varphi_1^{\textcolor{red}{1}}+a_2\varphi_2^{\textcolor{red}{1}}+a_3(\varphi_{3}^{\textcolor{red}{1}}+\varphi_3^{\textcolor{cyan}{2}})+a_4\varphi_4^{\textcolor{cyan}{2}}+a_5\varphi_5^{\textcolor{cyan}{2}}\]

\begin{center}
\includegraphics[scale=0.5]{figure1.png}
\includegraphics[scale=0.5]{figure2.png}
\end{center}

\end{document}
